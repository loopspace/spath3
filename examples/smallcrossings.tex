\documentclass[a4paper,11pt]{report}
%\url{https://tex.stackexchange.com/q/478294/86}
\usepackage{shellesc}
\usepackage{graphicx,amssymb,amstext,amsmath}
\usepackage{tikz}
\usetikzlibrary{
  decorations.pathreplacing,
  decorations.markings,
  hobby,
  knots,
  celtic,
  shapes.geometric,
  calc,
  external
}

\tikzexternalize

\begin{document}

\tikzset{
  knot diagram/every strand/.append style={
    line width=1.5pt,violet
  },
}

\tikzset{
  my style/.style={
    dashed
  }
}

Demonstration of the problem: the clipping circles are big enough that the circle associated with the crossing between the violet line and the dashed grey line overlaps the problematic crossing.

\tikzsetnextfilename{problem}
\begin{tikzpicture}[scale=0.9, use Hobby shortcut, add arrow/.style={postaction={decorate}, decoration={
  markings,
  mark=at position 0.25 with {\arrow[line width=1.5pt]{<}},
  mark=at position 0.653 with {\arrow[line width=1.5pt]{<}},
  mark=at position 0.75 with {\arrow[line width=1.5pt]{<}}}}]
\begin{knot}[
  consider self intersections=true,
  ignore endpoint intersections=false,
  flip crossing=3,
  background clip/.append style={
    preaction={
      fill=gray,
      fill opacity=.5,
    }
  },
  rotate=180]
  \strand ([closed]0,0) .. (0.7,-0.7) .. (0,-1) .. (-1.7,0) .. (-1.5,2.2) .. (0,2.5) .. (.5,2) .. (-.5,1) [add arrow,violet] .. (.5,0) .. (0,-.5) .. (-.5,0) .. (.5,1) .. (-.5,2) .. (-1.5,1) .. (0,0);
  \strand[knot diagram/only when rendering/.style={my style}, lightgray] (-0.7,-1.2) -- (-0.7,-.9); 
  \strand[knot diagram/only when rendering/.style={my style}, lightgray] (-0.7,-.75) -- (-0.7,2.5);
\end{knot}
\path (0,-.7);
\end{tikzpicture}

%---------------------------------------------------------------

Fix one: shrink the clipping circles using the \verb+clip radius+ key.

\tikzsetnextfilename{fix-one}
\begin{tikzpicture}[scale=0.9, use Hobby shortcut, add arrow/.style={postaction={decorate}, decoration={
  markings,
  mark=at position 0.25 with {\arrow[line width=1.5pt]{<}},
  mark=at position 0.653 with {\arrow[line width=1.5pt]{<}},
  mark=at position 0.75 with {\arrow[line width=1.5pt]{<}}}}]
\begin{knot}[
  consider self intersections=true,
  ignore endpoint intersections=false,
  flip crossing=3,
  clip radius=3pt,
  background clip/.append style={
    preaction={
      fill=gray,
      fill opacity=.5,
    }
  },
  rotate=180]
  \strand ([closed]0,0) .. (0.7,-0.7) .. (0,-1) .. (-1.7,0) .. (-1.5,2.2) .. (0,2.5) .. (.5,2) .. (-.5,1) [add arrow,violet] .. (.5,0) .. (0,-.5) .. (-.5,0) .. (.5,1) .. (-.5,2) .. (-1.5,1) .. (0,0);
  \strand[knot diagram/only when rendering/.style={my style}, lightgray] (-0.7,-1.2) -- (-0.7,-.9); 
  \strand[knot diagram/only when rendering/.style={my style}, lightgray] (-0.7,-.75) -- (-0.7,2.5);
\end{knot}
\path (0,-.7);
\end{tikzpicture}

Fix two: the dashed grey lines don't appear to actually be anything to do with the knot, so draw them separately.
This means that there aren't two crossings near to each other any more.

\tikzsetnextfilename{fix-two}
\begin{tikzpicture}[scale=0.9, use Hobby shortcut, add arrow/.style={postaction={decorate}, decoration={
  markings,
  mark=at position 0.25 with {\arrow[line width=1.5pt]{<}},
  mark=at position 0.653 with {\arrow[line width=1.5pt]{<}},
  mark=at position 0.75 with {\arrow[line width=1.5pt]{<}}}}]
\draw[
  line width=1.5pt,
  my style,
  lightgray,
  rotate=180
] (-0.7,-1.2) -- (-0.7,2.5);
\begin{knot}[
  consider self intersections=true,
  ignore endpoint intersections=false,
%  flip crossing=6,
  rotate=180]
  \strand ([closed]0,0) .. (0.7,-0.7) .. (0,-1) .. (-1.7,0) .. (-1.5,2.2) .. (0,2.5) .. (.5,2) .. (-.5,1) [add arrow,violet] .. (.5,0) .. (0,-.5) .. (-.5,0) .. (.5,1) .. (-.5,2) .. (-1.5,1) .. (0,0);
\end{knot}
\path (0,-.7);
\end{tikzpicture}

Fix three: you specify a lot of \emph{nodes} in your Hobby path.
If you aren't wedded to that specific path then reducing the number of nodes can simplify the knot algorithm.

\tikzsetnextfilename{fix-three}
\begin{tikzpicture}[scale=0.9, use Hobby shortcut, add arrow/.style={postaction={decorate}, decoration={
  markings,
  mark=at position 0.2 with {\arrow[line width=1.5pt]{<}},
  mark=at position 0.28 with {\arrow[line width=1.5pt]{<}},
  mark=at position 0.8 with {\arrow[line width=1.5pt]{<}}}}]
\draw[
  line width=1.5pt,
  my style,
  lightgray,
  rotate=180
] (-0.7,-2.2) -- (-0.7,3.5);
\begin{knot}[
  consider self intersections=true,
  end tolerance=3pt,
  flip crossing=4,
  rotate=180]
  \strand ([closed].5,2) [add arrow,violet] ..  (-.5,1) .. (.3,-.5) .. (-.3,-.5) .. (.5,1) .. ([tension=3]-1.5,1) .. (.7,-1) .. (-2,1);
\end{knot}
\path (0,-.7);
\end{tikzpicture}


\tikzsetnextfilename{large-scale}
\begin{tikzpicture}[scale=2, use Hobby shortcut, add arrow/.style={postaction={decorate}, decoration={
  markings,
  mark=at position 0.25 with {\arrow[line width=1.5pt]{<}},
  mark=at position 0.653 with {\arrow[line width=1.5pt]{<}},
  mark=at position 0.75 with {\arrow[line width=1.5pt]{<}}}}]
\begin{knot}[
  consider self intersections=true,
  ignore endpoint intersections=false,
  flip crossing=3,
  rotate=180]
  \strand ([closed]0,0) .. (0.7,-0.7) .. (0,-1) .. (-1.7,0) .. (-1.5,2.2) .. (0,2.5) .. (.5,2) .. (-.5,1) [add arrow,violet] .. (.5,0) .. (0,-.5) .. (-.5,0) .. (.5,1) .. (-.5,2) .. (-1.5,1) .. (0,0);
  \strand[knot diagram/only when rendering/.style={my style}, lightgray] (-0.7,-1.2) -- (-0.7,-.9); 
  \strand[knot diagram/only when rendering/.style={my style}, lightgray] (-0.7,-.75) -- (-0.7,2.5);
\end{knot}
\path (0,-.7);
\end{tikzpicture}

\end{document}
