% \iffalse meta-comment
%<*internal>
\iffalse
%</internal>
%<*readme>
----------------------------------------------------------------
calligraphy --- a style file for doing calligraphy with TikZ/PGF
E-mail: stacey@math.ntnu.no
Released under the LaTeX Project Public License v1.3c or later
See http://www.latex-project.org/lppl.txt
----------------------------------------------------------------

This package defines a way of simulating calligraphic lines with TikZ/PGF.
%</readme>
%<*internal>
\fi
\def\nameofplainTeX{plain}
\ifx\fmtname\nameofplainTeX\else
  \expandafter\begingroup
\fi
%</internal>
%<*install>
\input docstrip.tex
\keepsilent
\askforoverwritefalse
\preamble
----------------------------------------------------------------
calligraphy --- a style file for doing calligraphy with TikZ/PGF
E-mail: stacey@math.ntnu.no
Released under the LaTeX Project Public License v1.3c or later
See http://www.latex-project.org/lppl.txt
----------------------------------------------------------------

\endpreamble
\postamble

Copyright (C) 2011 by Andrew Stacey <stacey@math.ntnu.no>

This work may be distributed and/or modified under the
conditions of the LaTeX Project Public License (LPPL), either
version 1.3c of this license or (at your option) any later
version.  The latest version of this license is in the file:

http://www.latex-project.org/lppl.txt

This work is "maintained" (as per LPPL maintenance status) by
Andrew Stacey.

This work consists of the file  calligraphy.dtx
and the derived files           calligraphy.ins,
                                calligraphy.pdf, and
                                calligraphy.sty.

\endpostamble
\usedir{tex/latex/calligraphy}
\generate{
  \file{\jobname.sty}{\from{\jobname.dtx}{package}}
}
%</install>
%<install>\endbatchfile
%<*internal>
\usedir{source/latex/calligraphy}
\generate{
  \file{\jobname.ins}{\from{\jobname.dtx}{install}}
}
\nopreamble\nopostamble
\usedir{doc/latex/demopkg}
\generate{
  \file{README.txt}{\from{\jobname.dtx}{readme}}
}
\ifx\fmtname\nameofplainTeX
  \expandafter\endbatchfile
\else
  \expandafter\endgroup
\fi
%</internal>
%<*package>
\NeedsTeXFormat{LaTeX2e}
\ProvidesPackage{calligraphy}[2011/05/19 v1.0 Calligraphic commands with TikZ]
%</package>
%<*driver>
\documentclass{ltxdoc}
\usepackage[T1]{fontenc}
\usepackage{lmodern}
%\usepackage{morefloats}
\usepackage{tikz}
\usepackage{\jobname}
\usepackage[numbered]{hypdoc}
\definecolor{lstbgcolor}{rgb}{0.9,0.9,0.9} 
 
\usepackage{listings}
\lstloadlanguages{[LaTeX]TeX}
\lstset{breakatwhitespace=true,breaklines=true,language=TeX}
 
\usepackage{fancyvrb}

\newenvironment{example}
  {\VerbatimEnvironment
   \begin{VerbatimOut}[gobble=2]{example.out}}
  {\end{VerbatimOut}
   \begin{center}
%   \setlength{\parindent}{0pt}
   \fbox{\begin{minipage}{.9\linewidth}
     \lstset{breakatwhitespace=true,breaklines=true,language=TeX,basicstyle=\small}
     \lstinputlisting[]{example.out}
   \end{minipage}}

   \fbox{\begin{minipage}{.9\linewidth}
     \input{example.out}
   \end{minipage}}
\end{center}
}
\EnableCrossrefs
\CodelineIndex
\RecordChanges
\begin{document}
  \DocInput{\jobname.dtx}
\end{document}
%</driver>
% \fi
%
%
% \CharacterTable
%  {Upper-case    \A\B\C\D\E\F\G\H\I\J\K\L\M\N\O\P\Q\R\S\T\U\V\W\X\Y\Z
%   Lower-case    \a\b\c\d\e\f\g\h\i\j\k\l\m\n\o\p\q\r\s\t\u\v\w\x\y\z
%   Digits        \0\1\2\3\4\5\6\7\8\9
%   Exclamation   \!     Double quote  \"     Hash (number) \#
%   Dollar        \$     Percent       \%     Ampersand     \&
%   Acute accent  \'     Left paren    \(     Right paren   \)
%   Asterisk      \*     Plus          \+     Comma         \,
%   Minus         \-     Point         \.     Solidus       \/
%   Colon         \:     Semicolon     \;     Less than     \<
%   Equals        \=     Greater than  \>     Question mark \?
%   Commercial at \@     Left bracket  \[     Backslash     \\
%   Right bracket \]     Circumflex    \^     Underscore    \_
%   Grave accent  \`     Left brace    \{     Vertical bar  \|
%   Right brace   \}     Tilde         \~}
%
%
% \changes{1.0}{2011/05/03}{Converted to DTX file}
%
% \DoNotIndex{\newcommand,\newenvironment}
%
% \providecommand*{\url}{\texttt}
% \GetFileInfo{calligraphy.dtx}
% \title{The \textsf{calligraphy} package}
% \author{Andrew Stacey \\ \url{stacey@math.ntnu.no}}
% \date{\fileversion~from \filedate}
%
%
% \maketitle
%
% 
% \section{Introduction}
% 
% The \texttt{calligraphy} package is designed to enable calligraphic style drawings in TikZ.
% The idea is to be able to ``stroke'' a line with a ``pen''.
% As a simple example, compare the two lines in the following picture.
% 
% \begin{center}
% \begin{tikzpicture}
% \pen (-135:.25) -- (45:.25);
% \draw[line width=.5cm] (0,0) .. controls +(45:1) and +(-135:1) .. ++(3,0);
% \calligraphy (0,-1) .. controls +(45:1) and +(-135:1) .. ++(3,0);
% \end{tikzpicture}
% \end{center}
% 
% The paths are identical in definition but the first is drawn using the standard TikZ path with a line width of .5cm.
% The second is ``stroked'' with a calligraphic pen of width .5cm angled at 45 degrees.
% 
% \section{How It Works}
% 
% To know how to use this package, it is worth knowing a little about how it works.
% A ``pen'' is a path, as is the line that is the template for the pen stroke.
% The two paths are joined together to form a region which is filled.
% Thus in constructing the example given in the introduction, the following path is built.
% 
% \begin{center}
% \begin{tikzpicture}
% \draw (0,-1) .. controls +(45:1) and +(-135:1) .. node (op) {} ++(3,0) -- node (pp) {} ++(45:.5) .. controls +(-135:1) and +(45:1) .. node (opr) {} ++(-3,0) -- node (ppr) {} (0,-1);
% \draw[<-] (op) -- ++(0,-.5) node[anchor=north] {original path};
% \draw[<-] (pp) -- ++(.5,0) node[anchor=west] {pen path};
% \draw[<-] (opr) -- ++(0,.5) node[anchor=south] {original path reversed};
% \draw[<-] (ppr) -- ++(-.5,0) node[anchor=east] {pen path reversed};
% \end{tikzpicture}
% \end{center}
% 
% What is important to note about this is that the ``pen'' isn't \emph{actually} dragged along the path, it is merely a simulation.
% However, it should work as expected providing the following golden rule is not violated:
% 
% \begin{quotation}
% Never \emph{push} a calligraphic pen.
% \end{quotation}
% 
% This is good advice for ordinary calligraphy as well, so a path that is realisable as an honest calligraphic path should be fine with this package.
% 
% The paths for both pens and templates can be reasonably complicated.
% They can contain gaps, but should not contain closed paths, nor rectangles.
% The implementation works by breaking a path into its constituent pieces (broken up by ``move to''s) and working bit by bit.
% 
% \begin{center}
% \begin{tikzpicture}
% \pen (-135:.25) -- (-135:.125) (45:.125) -- (45:.25);
% \calligraphy (0,0) .. controls +(45:1) and +(-135:1) .. +(3,0) ++(1.5,0) .. controls +(-135:2) and +(45:2) .. +(0,-4)  (0,-4) .. controls +(45:1) and +(-135:1) .. +(3,0);
% \end{tikzpicture}
% \end{center}
% 
% \section{Copperplate}
% 
% Copperplate pens are somewhat special.
% They are ``thin'' so don't need the same treatment as a ``thick'' pen, but one should be able to vary the pressure with a copperplate pen to get a variation of thickness.
% Specifying a copperplate pen is straightforward: it is a pen with no thickness.
% 
% \begin{center}
% \begin{tikzpicture}[line width=2pt]
% \pen (0,0);
% \calligraphy (0,0) .. controls +(45:1) and +(-135:1) .. +(3,0) ++(1.5,0) .. controls +(-135:2) and +(45:2) .. +(0,-3)  (0,-3) .. controls +(45:1) and +(-135:1) .. +(3,0);
% \end{tikzpicture}
% \end{center}
% 
% With a copperplate pen, the segments of a path are tapered.
% Copperplate and normal pens can be mixed.
% Any part of the pen specification that has no length is treated as a copperplate pen.
% 
% \begin{center}
% \begin{tikzpicture}[line width=1pt]
% \pen (-135:.125) -- (0,0) (45:.125);
% \calligraphy (0,0) .. controls +(45:1) and +(-135:1) .. +(3,0) ++(1.5,0) .. controls +(-135:2) and +(45:2) .. +(0,-3)  (0,-3) .. controls +(45:1) and +(-135:1) .. +(3,0);
% \end{tikzpicture}
% \end{center}
% 
% \section{Annotations}
% 
% As an addition, calligraphic paths can be annotated to show how they were constructed.
% A path with an arrow is drawn at a set offset from the last part of the path.
% 
% \begin{center}
% \begin{tikzpicture}[line width=1pt]
% \pen (-135:.125) -- (0,0) (45:.125);
% \calligraphy[annotate] (0,0) .. controls +(45:1) and +(-135:1) .. +(3,0) ++(1.5,0) .. controls +(-135:2) and +(45:2) .. +(0,-3)  (0,-3) .. controls +(45:1) and +(-135:1) .. +(3,0);
% \end{tikzpicture}
% \end{center}
% 
% \section{Style Options}
% 
% There are plenty of options for styling the paths and pens.
% 
% \subsection{Definition Options}
% 
% The \texttt{define pen} option is used to define a pen.
% It can be given an argument which will be the name of the pen.
% Pens are stored globally, and while they need a \texttt{tikz} or \texttt{tikzpicture} to be defined, they can be used in other pictures (it would also be possible to save the definitions to a file and load them in).
% The \texttt{pen} command is a shortcut for a path with the \texttt{define pen} option set.
% The name can also be specified by the \texttt{pen name} option.
% 
% To use a pen, we use the option \texttt{use pen}.
% Technically, this works using a preaction, so the path itself should not be stroked or filled.
% The \texttt{calligraphy} command is a shortcut for a path with the \texttt{use pen} option set.
% The \texttt{pen name} can be used to specify a particular pen.
% 
% \subsection{Style Options}
% 
% Since calligraphic paths are sometimes filled and sometimes stroked, the \texttt{pen colour} option is provided to ensure that the colours are set correctly.
% 
% It is possible to style particular ``nibs'' (i.e., segments of the pen path) separately.
% This is the \texttt{pen nib} option, which takes two arguments.
% The first is the index of the part of the nib and the second is the style options to be applied.
% 
% It is also possible to style particular parts of the template path.
% This is done by putting \texttt{[line style={}]} in the template path at the relevant part.
% The style is saved and applied to that segment of the template path.
% 
% The tapering of copperplate paths can be controled by the \texttt{taper} option.
% It takes arguments \texttt{none}, \texttt{both}, \texttt{start}, and \texttt{end}.
% 
% The style of the annotations is controlled by the options \texttt{annotation style} (for global style), \texttt{annotation shift} which controls how the annotation path is shifted from the final component of the calligraphic path, \texttt{annotation nodes style} for global style of the nodes, \texttt{annotation node style} for styling of particular nodes.
% \begin{center}
% \begin{tikzpicture}
% \calligraphy[pen colour=green,pen nib={2}{color=red}] (0,0) .. controls +(45:1) and +(-135:1) .. +(3,0) ++(1.5,0) .. controls +(-135:2) and +(45:2) .. +(0,-3)  (0,-3) .. controls +(45:1) and +(-135:1) .. +(3,0);
% \calligraphy[annotate,annotation node style={2}{circle,draw,fill=white}] (4,0) .. controls +(45:1) and +(-135:1) .. +(3,0) ++(1.5,0) .. controls +(-135:2) and +(45:2) .. +(0,-3)  (4,-3) .. controls +(45:1) and +(-135:1) .. +(3,0);
% \calligraphy[pen nib style={line width=1pt}] (0,-4) .. controls +(45:1) and +(-135:1) .. +(3,0) ++(1.5,0) .. controls +(-135:2) and +(45:2) .. +(0,-3)  (0,-7) .. controls +(45:1) and +(-135:1) .. +(3,0);
% \calligraphy[annotate,annotation style={dashed,->}] (4,-4) .. controls +(45:1) and +(-135:1) .. +(3,0) ++(1.5,0) .. controls +(-135:2) and +(45:2) .. +(0,-3)  (4,-7) .. controls +(45:1) and +(-135:1) .. +(3,0);
% \end{tikzpicture}
% \end{center}
% 
%
% \StopEventually{}
%
% \section{Implementation}
%
% \iffalse
%<*package>
% \fi
%
%    \begin{macrocode}
\ProvidesPackage{calligraphy}
%    \end{macrocode}
%
% \begin{macro}{\ge@addto@macro}
%    \begin{macrocode}
\long\def\ge@addto@macro#1#2{%
  \begingroup
  \toks@\expandafter\expandafter\expandafter{\expandafter#1#2}%
  \xdef#1{\the\toks@}%
  \endgroup}
%    \end{macrocode}
% \end{macro}
%
%
% \begin{macro}{\ge@addbefore@macro}
%    \begin{macrocode}
\long\def\ge@addbefore@macro#1#2{%
  \begingroup
  \toks@\expandafter\expandafter\expandafter{\expandafter#2#1}%
  \xdef#1{\the\toks@}%
  \endgroup}
%    \end{macrocode}
% \end{macro}
%
%
% \begin{macro}{\g@addbefore@macro}
%    \begin{macrocode}
\long\def\g@addbefore@macro#1#2{%
  \begingroup
    \toks@\expandafter{\expandafter#2#1}%
    \xdef#1{\the\toks@}%
  \endgroup}
\newif\ifcalligraphy@annotate
\newif\ifspath@taper@start
\newif\ifspath@taper@end
\spath@taper@starttrue
\spath@taper@endtrue
\newcounter{calligraphy@label}
\newcounter{calligraphy@nib}
\newcounter{calligraphy@path@parts}
\pgfkeys{
  /tikz/irrelevant for bounding box/.code={
    \pgf@relevantforpicturesizefalse
  },
  /tikz/calligraphy/.style={
    /pgf/calligraphy/.cd
  },
  /pgf/calligraphy/.search also={/pgf,/tikz},
  /pgf/calligraphy/.cd,
  define pen/.style={
    /tikz/irrelevant for bounding box,
    save pen,
    pen name=#1,
  },
  pen name/.initial=default,
  save pen/.code={
    \tikz@addmode{\pgfsyssoftpath@getcurrentpath\calligraphy@pen@default
    \expandafter\global\expandafter\let\csname calligraphy@pen@\pgfkeysvalueof{/pgf/calligraphy/pen name}\endcsname=\calligraphy@pen@default}
  },
  use pen/.style={
    pen name=#1,
    /tikz/preaction={/tikz/save path=\calligraphy@main@path},
    /tikz/preaction={/pgf/calligraphy/stroke with calligraphy pen}
  },
  stroke with calligraphy pen/.code={
    \expandafter\let\expandafter\calligraphy@pen@path\csname calligraphy@pen@\pgfkeysvalueof{/pgf/calligraphy/pen name}\endcsname
    \cgrphy@thickenpath{\calligraphy@main@path}{\calligraphy@pen@path}
  },
  pen colour/.initial=black,
  annotate if/.is if=calligraphy@annotate,
  annotate/.style={
    annotate if,
    annotate reset
  },
  annotate reset/.code={
    \setcounter{calligraphy@label}{0}
  },
  annotation style/.initial={->},
  annotation shift/.initial={(0,1ex)},
  annotation nodes style/.initial={anchor=south east},
  pen nib/.code 2 args={
    \expandafter\def\csname calligraphy@nib@style@#1\endcsname{#2}
  },
  annotation node style/.code 2 args={
    \expandafter\def\csname calligraphy@annotation@style@#1\endcsname{#2}
  },
  pen nib style/.initial={},
  taper/.is choice,
  taper/none/.style={
    taper at start=false,
    taper at end=false,
  },
  taper/both/.style={
    taper at start=true,
    taper at end=true,
  },
  taper/start/.style={
    taper at start=true,
    taper at end=false,
  },
  taper/end/.style={
    taper at start=false,
    taper at end=true,
  },
  taper at start/.is if=spath@taper@start,
  taper at end/.is if=spath@taper@end,
  line style/.code={
    \pgfsyssoftpath@getcurrentpath\spath@tmp@path
    \spath@num@components\spath@tmp@path
    \expandafter\def\csname calligraphy@part@style@\arabic{spath@length}\endcsname{#1}
  },
}
%    \end{macrocode}
% \end{macro}
%
%
% \begin{macro}{\calligraphystyle}
%    \begin{macrocode}
\newcommand{\calligraphystyle}[1]{\pgfkeys{/pgf/calligraphy/.cd,#1}}
%    \end{macrocode}
% \end{macro}
%
%
% \begin{macro}{\pen}
%    \begin{macrocode}
\newcommand{\pen}[1][]{\path[calligraphy,define pen,#1]}
%    \end{macrocode}
% \end{macro}
%
%
% \begin{macro}{\calligraphy}
%    \begin{macrocode}
\newcommand{\calligraphy}[1][]{\path[calligraphy,use pen,#1]}
%    \end{macrocode}
% \end{macro}
%
%
% \begin{macro}{\definepen}
%    \begin{macrocode}
\newcommand{\definepen}[1][]{\tikz \path[calligraphy,define pen,#1]}
%    \end{macrocode}
% \end{macro}
%
% Calligraphy commands
%
% Thicken a piece of the path with the pen
% So we step through the pen path, thickening
%
%
% \begin{macro}{\cgrphy@thickenpartialpath}
%    \begin{macrocode}
\def\cgrphy@thickenpartialpath#1#2{%
  \pgfsys@beginscope
  \def\thick@path{}%
  \setcounter{calligraphy@nib}{0}
  \def\thick@action##1##2##3{%
    \edef\this@action{\string##1}%
    \ifx\this@action\spath@moveto
    \ifx\thick@path\@empty
    \else
    \stepcounter{calligraphy@nib}
     \spath@length{\thick@path}
     \ifnum\value{spath@length}=1\relax
      \cgrphy@thickenpartialpathwithstroke{#1}{\thick@path}
     \else
      \cgrphy@thickenpartialpathwithfill{#1}{\thick@path}
     \fi
     \def\thick@path{}%
    \fi
    \fi
    \def\spath@to@add{##1{##2}{##3}}
  }
  \let\thick@append=\ge@addto@macro
  \def\@temp{\spath@walk\thick@path\thick@action\thick@append}
  \expandafter\@temp#2\relax
    \stepcounter{calligraphy@nib}
  \spath@length{\thick@path}
  \ifnum\value{spath@length}=1\relax
  \cgrphy@thickenpartialpathwithstroke{#1}{\thick@path}
  \else
  \cgrphy@thickenpartialpathwithfill{#1}{\thick@path}
  \fi
  \ifcalligraphy@annotate
  \cgrphy@annotatepath{#1}{\thick@path}
  \fi
  \pgfsys@endscope
}
%    \end{macrocode}
% \end{macro}
%
%
% \begin{macro}{\cgrphy@thickenpartialpathwithfill}
%    \begin{macrocode}
\def\cgrphy@thickenpartialpathwithfill#1#2{%
  \pgfsys@beginscope
  \let\tikz@options=\pgfutil@empty
  \expandafter\let\expandafter\calligraphy@nib@style\csname calligraphy@nib@style@\arabic{calligraphy@nib}\endcsname
  \ifx\calligraphy@nib@style\relax
  \else
   \expandafter\calligraphystyle\expandafter{\calligraphy@nib@style}
  \fi
  \expandafter\let\expandafter\calligraphy@part@style\csname calligraphy@part@style@\arabic{calligraphy@path@parts}\endcsname
  \ifx\calligraphy@part@style\relax
  \else
   \expandafter\calligraphystyle\expandafter{\calligraphy@part@style}
  \fi
  \tikz@options
  \spath@start{#2}
  \let\pen@sx=\spath@xlen
  \let\pen@sy=\spath@ylen
  \spath@end{#2}
  \let\pen@ex=\spath@xlen
  \let\pen@ey=\spath@ylen
  \spath@start{#1}
  \let\path@sx=\spath@xlen
  \let\path@sy=\spath@ylen
  \spath@end{#1}
  \let\path@ex=\spath@xlen
  \let\path@ey=\spath@ylen
  \spath@translate{#1}{\pen@sx}{\pen@sy}%
  \let\lower@path=\spath@tr@path
  \spath@translate{#2}{\path@ex}{\path@ey}
  \let\right@path=\spath@tr@path
  \spath@reverse{#1}
  \spath@translate{\spath@re@path}{\pen@ex}{\pen@ey}
  \let\upper@path=\spath@tr@path
  \spath@reverse{#2}
  \spath@translate{\spath@re@path}{\path@sx}{\path@sy}
  \let\left@path=\spath@tr@path
  \spath@cat{\lower@path}{\right@path}
  \spath@cat{\spath@catpath}{\upper@path}
  \spath@cat{\spath@catpath}{\left@path}
  \def\spath@to@add{\pgfsyssoftpath@closepath{0pt}{0pt}}
  \ge@addto@macro\spath@catpath\spath@to@add
  \pgfsyssoftpath@setcurrentpath{\spath@catpath}
  \pgfsyssoftpath@flushcurrentpath
  \pgfusepath{fill}
  \pgfsys@endscope
}
%    \end{macrocode}
% \end{macro}
%
%
% \begin{macro}{\cgrphy@thickenpartialpathwithstroke}
%    \begin{macrocode}
\def\cgrphy@thickenpartialpathwithstroke#1#2{%
  \pgfsys@beginscope
  \let\tikz@options=\pgfutil@empty
  \pgfkeysgetvalue{/pgf/calligraphy/pen nib style}{\calligraphy@nib@style}
  \ifx\calligraphy@nib@style\relax
  \else
   \expandafter\calligraphystyle\expandafter{\calligraphy@nib@style}
  \fi
  \expandafter\let\expandafter\calligraphy@nib@style\csname calligraphy@nib@style@\arabic{calligraphy@nib}\endcsname
  \ifx\calligraphy@nib@style\relax
  \else
   \expandafter\calligraphystyle\expandafter{\calligraphy@nib@style}
  \fi
  \expandafter\let\expandafter\calligraphy@part@style\csname calligraphy@part@style@\arabic{calligraphy@path@parts}\endcsname
  \ifx\calligraphy@part@style\relax
  \else
   \expandafter\calligraphystyle\expandafter{\calligraphy@part@style}
  \fi
  \tikz@options
  \spath@start{#2}
  \let\pen@sx=\spath@xlen
  \let\pen@sy=\spath@ylen
  \spath@translate{#1}{\pen@sx}{\pen@sy}%
  \spath@split\spath@tr@path
  \spath@taper\spath@split@path
  \pgfsys@endscope
}
%    \end{macrocode}
% \end{macro}
%
%
% \begin{macro}{\cgrphy@annotatepath}
%    \begin{macrocode}
\def\cgrphy@annotatepath#1#2{%
  \pgfsys@beginscope
  \spath@start{#2}
  \let\pen@sx=\spath@xlen
  \let\pen@sy=\spath@ylen
  \spath@translate{#1}{\pen@sx}{\pen@sy}%
  \let\spath@tmp@path=\spath@tr@path
  \path[irrelevant for bounding box] \pgfkeysvalueof{/pgf/calligraphy/annotation shift};
  \pgfgetlastxy{\pen@sx}{\pen@sy}
  \spath@translate{\spath@tmp@path}{\pen@sx}{\pen@sy}%
  \pgfkeysgetvalue{/pgf/calligraphy/annotation style}{\cgrphy@annotation@style}
  \let\tikz@options=\pgfutil@empty
  \ifx\cgrphy@annotation@style\relax
  \else
   \expandafter\calligraphystyle\expandafter{\cgrphy@annotation@style}
  \fi
  \tikz@options
  \spath@reprocess{\spath@tr@path}
  \spath@rep@path
  \pgfusepath{stroke}
  \stepcounter{calligraphy@label}
  \spath@end{\spath@tr@path}
  \pgfkeysgetvalue{/pgf/calligraphy/annotation nodes style}{\cgrphy@node@options}
  \ifx\cgrphy@node@options\relax
  \else
   \g@addto@macro\cgrphy@node@options{,}
  \fi
  \expandafter\let\expandafter\cgrphy@node@style\csname calligraphy@annotation@style@\arabic{calligraphy@label}\endcsname
  \ifx\cgrphy@node@style\relax
  \else
   \ge@addto@macro{\cgrphy@node@options}{\cgrphy@node@style}
  \fi
  \ifx\cgrphy@node@options\relax
  \let\cgrphy@node@options=\pgfutil@empty
  \fi
  \expandafter\node\expandafter[\cgrphy@node@options] at (\spath@xlen,\spath@ylen) {\arabic{calligraphy@label}};
  \pgfsys@endscope
}
%    \end{macrocode}
% \end{macro}
%
%
% \begin{macro}{\cgrphy@thickenpath}
%    \begin{macrocode}
\def\cgrphy@thickenpath#1#2{%
  \pgfsys@beginscope
  \pgfsetstrokecolor{\pgfkeysvalueof{/pgf/calligraphy/pen colour}}
  \pgfsetfillcolor{\pgfkeysvalueof{/pgf/calligraphy/pen colour}}
  \def\th@path{}%
  \setcounter{calligraphy@path@parts}{-1}
  \def\th@action##1##2##3{%
    \edef\this@action{\string##1}
    \ifx\this@action\spath@moveto
     \ifx\th@path\@empty
    \else
      \cgrphy@thickenpartialpath\th@path{#2}
     \fi
     \stepcounter{calligraphy@path@parts}
    \def\th@path{}
    \fi
    \def\spath@to@add{##1{##2}{##3}}
  }
  \let\th@append=\ge@addto@macro
  \def\@temp{\spath@walk\th@path\th@action\th@append}
  \expandafter\@temp#1\relax
  \cgrphy@thickenpartialpath\th@path{#2}
  \pgfsys@endscope
}
%    \end{macrocode}
% \end{macro}
%
% Soft path manipulation commands
%
%
% \begin{macro}{\spath@cat}
%    \begin{macrocode}
\def\spath@cat#1#2{
  \let\spath@catpath=#1
  \expandafter\spath@trimfirst#2\relax
  \ge@addto@macro\spath@catpath\spath@trimmed@path
}
%    \end{macrocode}
% \end{macro}
%
%
% \begin{macro}{\spath@trimfirst}
%    \begin{macrocode}
\def\spath@trimfirst#1#2#3#4\relax{%
  \edef\this@action{\string#1}%
  \ifx\this@action\spath@moveto
  \def\spath@trimmed@path{#4}%
  \else
  \def\spath@trimmed@path{#1{#2}{#3}#4}%
  \fi
}
%    \end{macrocode}
% \end{macro}
%
%
% \begin{macro}{\spath@walk}
%    \begin{macrocode}
\def\spath@walk#1#2#3#4{
  \let\spath@path@cmd=#4%
  \ifx\spath@path@cmd\relax
  \let\spath@next@action=\@gobblefour
  \else
  \let\spath@next@action=\spath@modifypath
  \fi
  \spath@next@action{#1}{#2}{#3}{#4}%
}
%    \end{macrocode}
% \end{macro}
%
%
% \begin{macro}{\spath@modifypath}
%    \begin{macrocode}
\def\spath@modifypath#1#2#3#4#5#6{%
  #2{#4}{#5}{#6}
  #3#1\spath@to@add
  \spath@walk{#1}{#2}{#3}}
\newcounter{spath@length}
%    \end{macrocode}
% \end{macro}
%
%
% \begin{macro}{\spath@length}
%    \begin{macrocode}
\def\spath@length#1{%
  \def\spath@len@path{}%
  \setcounter{spath@length}{0}%
  \def\spath@len@action##1##2##3{%
    \stepcounter{spath@length}%
    \edef\spath@to@add{}%
  }
  \let\spath@len@append=\ge@addto@macro
  \def\@temp{\spath@walk\spath@len@path\spath@len@action\spath@len@append}
  \expandafter\@temp#1\relax
}
%    \end{macrocode}
% \end{macro}
%
%
% \begin{macro}{\spath@real@length}
%    \begin{macrocode}
\def\spath@real@length#1{%
  \def\spath@rlen@path{}%
  \setcounter{spath@length}{0}%
  \def\spath@rlen@action##1##2##3{%
    \edef\spath@this@rlen@action{\string##1}
    \ifx\spath@this@rlen@action\spath@lineto
     \stepcounter{spath@length}%
    \else
    \ifx\spath@this@rlen@action\spath@curveto
     \stepcounter{spath@length}%
    \fi
    \fi
    \edef\spath@to@add{}%
  }
  \let\spath@rlen@append=\ge@addto@macro
  \def\@temp{\spath@walk\spath@rlen@path\spath@rlen@action\spath@rlen@append}
  \expandafter\@temp#1\relax
}
%    \end{macrocode}
% \end{macro}
%
%
% \begin{macro}{\spath@num@components}
%    \begin{macrocode}
\def\spath@num@components#1{%
  \def\spath@cmpts@path{}%
  \setcounter{spath@length}{0}%
  \def\spath@cmpts@action##1##2##3{%
    \edef\spath@this@cmpts@action{\string##1}
    \ifx\spath@this@cmpts@action\spath@moveto
     \stepcounter{spath@length}%
    \fi
    \edef\spath@to@add{}%
  }
  \let\spath@cmpts@append=\ge@addto@macro
  \def\@temp{\spath@walk\spath@cmpts@path\spath@cmpts@action\spath@cmpts@append}
  \expandafter\@temp#1\relax
}
%    \end{macrocode}
% \end{macro}
%
%
% \begin{macro}{\spath@translate}
%    \begin{macrocode}
\def\spath@translate#1#2#3{%
  \def\spath@tr@path{}%
  \def\spath@tr@action##1##2##3{%
    \pgfmathsetmacro{\spath@tr@xlen}{##2+#2}
    \pgfmathsetmacro{\spath@tr@ylen}{##3+#3}
    \edef\spath@to@add{\noexpand##1{\spath@tr@xlen pt}{\spath@tr@ylen pt}}
  }
  \let\spath@tr@append=\ge@addto@macro
  \def\@temp{\spath@walk\spath@tr@path\spath@tr@action\spath@tr@append}
  \expandafter\@temp#1\relax
}
%    \end{macrocode}
% \end{macro}
%
%
% \begin{macro}{\spath@trimlast}
%    \begin{macrocode}
\def\spath@trimlast#1\pgfsyssoftpath@movetotoken\relax{\def\spath@trimmed{#1}}
%    \end{macrocode}
% \end{macro}
%
%    \begin{macrocode}
\foreach \spath@cpt in {
  moveto,
  lineto,
  curvetosupporta,
  curvetosupportb,
  curveto,
  rectcorner,
  rectsize%
} {
\expandafter\xdef\csname spath@\spath@cpt\endcsname{\expandafter\string\csname pgfsyssoftpath@\spath@cpt token\endcsname}
}
%    \end{macrocode}
%
% \begin{macro}{\spath@closepath}
%    \begin{macrocode}
\edef\spath@closepath{\string\pgfsyssoftpath@closepath}
%    \end{macrocode}
% \end{macro}
%
%
% \begin{macro}{\spath@reprocess}
%    \begin{macrocode}
\def\spath@reprocess#1{%
  \def\spath@rep@path{}%
  \def\spath@rep@action##1##2##3{%
    \edef\spath@this@rep@action{\string##1}
    \ifx\spath@this@rep@action\spath@moveto
     \edef\spath@to@add{\noexpand\pgfpathmoveto{\noexpand\pgfpoint{##2}{##3}}}
    \else
    \ifx\spath@this@rep@action\spath@lineto
     \edef\spath@to@add{\noexpand\pgfpathlineto{\noexpand\pgfpoint{##2}{##3}}}
    \else
    \ifx\spath@this@rep@action\spath@curvetosupporta
     \edef\spath@to@add{\noexpand\pgfpathcurveto{\noexpand\pgfpoint{##2}{##3}}}
    \else
    \ifx\spath@this@rep@action\spath@curvetosupportb
     \edef\spath@to@add{{\noexpand\pgfpoint{##2}{##3}}}
    \else
    \ifx\spath@this@rep@action\spath@curveto
     \edef\spath@to@add{{\noexpand\pgfpoint{##2}{##3}}}
    \else
    \ifx\spath@this@rep@action\spath@closepath
     \edef\spath@to@add{\noexpand\pgfpathclose{\noexpand\pgfpoint{##2}{##3}}}
    \fi
    \fi
    \fi
    \fi
    \fi
    \fi
  }
  \let\spath@rep@append=\ge@addto@macro
  \def\@temp{\spath@walk\spath@rep@path\spath@rep@action\spath@rep@append}
  \expandafter\@temp#1\relax
}
%    \end{macrocode}
% \end{macro}
%
%
% \begin{macro}{\spath@reverse}
%    \begin{macrocode}
\def\spath@reverse#1{%
  \def\spath@re@path{}%
  \def\spath@re@action##1##2##3{%
    \edef\spath@this@re@action{\string##1}
    \ifx\spath@this@re@action\spath@curvetosupporta
     \edef\spath@to@add{\noexpand\pgfsyssoftpath@curvetosupportbtoken{##2}{##3}\noexpand\pgfsyssoftpath@curvetotoken}
    \else
     \ifx\spath@this@re@action\spath@curvetosupportb
      \edef\spath@to@add{\noexpand\pgfsyssoftpath@curvetosupportatoken{##2}{##3}}
     \else
      \ifx\spath@this@re@action\spath@curveto
       \edef\spath@to@add{{##2}{##3}}
      \else
       \edef\spath@to@add{{##2}{##3}\noexpand##1}
      \fi
     \fi
    \fi
    }
  \let\spath@re@append=\ge@addbefore@macro
  \def\@temp{\spath@walk\spath@re@path\spath@re@action\spath@re@append}
  \expandafter\@temp#1\relax
  \expandafter\spath@trimlast\spath@re@path\relax
  \let\spath@re@path=\spath@trimmed
  \g@addbefore@macro\spath@re@path\pgfsyssoftpath@movetotoken
}
%    \end{macrocode}
% \end{macro}
%
%
% \begin{macro}{\spath@start}
%    \begin{macrocode}
\def\spath@start#1{%
  \def\spath@st@path{}%
  \def\spath@st@action##1##2##3{%
    \def\spath@xlen{##2}%
    \def\spath@ylen{##3}%
    \def\spath@st@action####1####2####3{}
  }
  \let\spath@st@append=\ge@addto@macro
  \def\@temp{\spath@walk\spath@st@path\spath@st@action\spath@st@append}
  \expandafter\@temp#1\relax
}
%    \end{macrocode}
% \end{macro}
%
%
% \begin{macro}{\spath@end}
%    \begin{macrocode}
\def\spath@end#1{%
  \def\spath@end@path{}%
  \def\spath@end@action##1##2##3{%
    \def\spath@xlen{##2}%
    \def\spath@ylen{##3}%
  }
  \let\spath@end@append=\ge@addto@macro
  \def\@temp{\spath@walk\spath@end@path\spath@end@action\spath@end@append}
  \expandafter\@temp#1\relax
}
%    \end{macrocode}
% \end{macro}
%
%
% \begin{macro}{\spath@first}
%    \begin{macrocode}
\def\spath@first#1{%
  \expandafter\spath@@first#1\relax
}
%    \end{macrocode}
% \end{macro}
%
%
% \begin{macro}{\spath@@first}
%    \begin{macrocode}
\def\spath@@first#1#2#3#4#5#6#7\relax{%
  \def\spath@start@path{#1{#2}{#3}#4{#5}{#6}}%
  \def\spath@rest@path{#1}
  \edef\spath@first@action{\string#4}%
  \ifx\spath@first@action\spath@curvetosupporta
   \spath@first@curveto#7\relax
  \else
   \g@addto@macro{\spath@rest@path}{{#5}{#6}#7}
  \fi
}
%    \end{macrocode}
% \end{macro}
%
%
% \begin{macro}{\spath@first@curveto}
%    \begin{macrocode}
\def\spath@first@curveto#1#2#3#4#5#6#7\relax{%
  \g@addto@macro{\spath@rest@path}{{#5}{#6}#7}%
  \g@addto@macro{\spath@start@path}{#1{#2}{#3}#4{#5}{#6}}
}
%    \end{macrocode}
% \end{macro}
%
%
% \begin{macro}{\spath@taper@out}
%    \begin{macrocode}
\def\spath@taper@out#1{%
  \expandafter\spath@find@taper@type#1\relax
  \ifx\spath@taper@type\spath@lineto
   \expandafter\spath@taper@lineto@out#1\relax
  \else
  \ifx\spath@taper@type\spath@curvetosupporta
   \expandafter\spath@taper@curveto@out#1\relax
  \fi
  \fi
}
%    \end{macrocode}
% \end{macro}
%
%
% \begin{macro}{\spath@taper@in}
%    \begin{macrocode}
\def\spath@taper@in#1{%
  \expandafter\spath@find@taper@type#1\relax
  \ifx\spath@taper@type\spath@lineto
   \expandafter\spath@taper@lineto@in#1\relax
  \else
  \ifx\spath@taper@type\spath@curvetosupporta
   \expandafter\spath@taper@curveto@in#1\relax
  \fi
  \fi
}
%    \end{macrocode}
% \end{macro}
%
%
% \begin{macro}{\spath@find@taper@type}
%    \begin{macrocode}
\def\spath@find@taper@type#1#2#3#4#5#6#7\relax{%
  \edef\spath@taper@type{\string#4}%
}
%    \end{macrocode}
% \end{macro}
%
%
% \begin{macro}{\spath@taper@lineto@out}
%    \begin{macrocode}
\def\spath@taper@lineto@out#1#2#3#4#5#6#7\relax{%
% #1 is \pgfsyssoftpath@movetotoken
% #2 is x-coord of starting point
% #3 is y-coord of starting point
% #4 is \pgfsyssoftpath@linetotoken
% #5 is x-coord of ending point
% #6 is y-coord of ending point
% #7 shouldn't have anything in
  \pgfmathsetmacro{\spath@sx}{.7 * #2 + .3 * #5}
  \pgfmathsetmacro{\spath@sy}{.7 * #3 + .3 * #6}
  \pgfmathsetmacro{\spath@ex}{.3 * #2 + .7 * #5}
  \pgfmathsetmacro{\spath@ey}{.3 * #3 + .7 * #6}
  \edef\spath@tmp{\noexpand\spath@taper@path{#2}{#3}{\spath@sx pt}{\spath@sy pt}{\spath@ex pt}{\spath@ey pt}{#5}{#6}}
  \spath@tmp
}
%    \end{macrocode}
% \end{macro}
%
%
% \begin{macro}{\spath@taper@curveto@out}
%    \begin{macrocode}
\def\spath@taper@curveto@out\pgfsyssoftpath@movetotoken#1#2\pgfsyssoftpath@curvetosupportatoken#3#4\pgfsyssoftpath@curvetosupportbtoken#5#6\pgfsyssoftpath@curvetotoken#7#8#9\relax{%
  \spath@taper@path{#1}{#2}{#3}{#4}{#5}{#6}{#7}{#8}
}
%    \end{macrocode}
% \end{macro}
%
%
% \begin{macro}{\spath@taper@lineto@in}
%    \begin{macrocode}
\def\spath@taper@lineto@in#1#2#3#4#5#6#7\relax{%
% #1 is \pgfsyssoftpath@movetotoken
% #2 is x-coord of starting point
% #3 is y-coord of starting point
% #4 is \pgfsyssoftpath@linetotoken
% #5 is x-coord of ending point
% #6 is y-coord of ending point
% #7 shouldn't have anything in
  \pgfmathsetmacro{\spath@sx}{.7 * #2 + .3 * #5}
  \pgfmathsetmacro{\spath@sy}{.7 * #3 + .3 * #6}
  \pgfmathsetmacro{\spath@ex}{.3 * #2 + .7 * #5}
  \pgfmathsetmacro{\spath@ey}{.3 * #3 + .7 * #6}
  \edef\spath@tmp{\noexpand\spath@taper@path{#5}{#6}{\spath@ex pt}{\spath@ey pt}{\spath@sx pt}{\spath@sy pt}{#2}{#3}}
  \spath@tmp
}
%    \end{macrocode}
% \end{macro}
%
%
% \begin{macro}{\spath@taper@curveto@in}
%    \begin{macrocode}
\def\spath@taper@curveto@in\pgfsyssoftpath@movetotoken#1#2\pgfsyssoftpath@curvetosupportatoken#3#4\pgfsyssoftpath@curvetosupportbtoken#5#6\pgfsyssoftpath@curvetotoken#7#8#9\relax{%
  \spath@taper@path{#7}{#8}{#5}{#6}{#3}{#4}{#1}{#2}
}
%    \end{macrocode}
% \end{macro}
%
%
% \begin{macro}{\spath@taper@path}
%    \begin{macrocode}
\def\spath@taper@path#1#2#3#4#5#6#7#8{
% #1 is x-coord of starting point
% #2 is y-coord of starting point
% #3 is x-coord of first control point
% #4 is y-coord of first control point
% #5 is x-coord of second control point
% #6 is y-coord of second control point
% #7 is x-coord of ending point
% #8 is y-coord of ending point
%
  \edef\spath@sx{#1}
  \edef\spath@sy{#2}
  \edef\spath@cx{#3}
  \edef\spath@cy{#4}
  \edef\spath@dx{#5}
  \edef\spath@dy{#6}
  \edef\spath@ex{#7}
  \edef\spath@ey{#8}
  
%    \end{macrocode}
% Orthogonal vector
%    \begin{macrocode}
  \pgfmathsetmacro{\spath@ox}{-\spath@ey + \spath@dy}
  \pgfmathsetmacro{\spath@oy}{\spath@ex - \spath@dx}
  
%    \end{macrocode}
% Adjust length to half the line width
%    \begin{macrocode}
  \pgfmathsetmacro{\spath@oox}{.5*\spath@ox * \pgflinewidth / veclen(\spath@ox,\spath@oy)}
  \pgfmathsetmacro{\spath@ooy}{.5*\spath@oy * \pgflinewidth / veclen(\spath@ox,\spath@oy)}
%    \end{macrocode}
% Shift the end point and the control points
%    \begin{macrocode}
  \pgfmathsetmacro{\spath@dx}{\spath@dx + \spath@oox}
  \pgfmathsetmacro{\spath@dy}{\spath@dy + \spath@ooy}
%    \end{macrocode}
% End points are shifted slightly to overlap the next component
%    \begin{macrocode}
  \pgfmathsetmacro{\spath@ex}{\spath@ex + \spath@oox + .1*\spath@ooy}
  \pgfmathsetmacro{\spath@ey}{\spath@ey + \spath@ooy - .1*\spath@oox}
%    \end{macrocode}
% Add the first pieces
%    \begin{macrocode}
  \let\spath@tapered@path=\pgfutil@empty
  \edef\spath@to@add{\noexpand\pgfsyssoftpath@movetotoken{\spath@sx pt}{\spath@sy pt}}
  \ge@addto@macro\spath@tapered@path\spath@to@add
  \edef\spath@to@add{\noexpand\pgfsyssoftpath@curvetosupportatoken{\spath@cx pt}{\spath@cy pt}}
  \ge@addto@macro\spath@tapered@path\spath@to@add
  \edef\spath@to@add{\noexpand\pgfsyssoftpath@curvetosupportbtoken{\spath@dx pt}{\spath@dy pt}}
  \ge@addto@macro\spath@tapered@path\spath@to@add
  \edef\spath@to@add{\noexpand\pgfsyssoftpath@curvetotoken{\spath@ex pt}{\spath@ey pt}}
  \ge@addto@macro\spath@tapered@path\spath@to@add
%    \end{macrocode}
% Shift the end points and the control points again
%    \begin{macrocode}
  \pgfmathsetmacro{\spath@dx}{\spath@dx - 2*\spath@oox}
  \pgfmathsetmacro{\spath@dy}{\spath@dy - 2*\spath@ooy}
  \pgfmathsetmacro{\spath@ex}{\spath@ex - 2*\spath@oox}
  \pgfmathsetmacro{\spath@ey}{\spath@ey - 2*\spath@ooy}
  \edef\spath@to@add{\noexpand\pgfsyssoftpath@linetotoken{\spath@ex pt}{\spath@ey pt}}
  \ge@addto@macro\spath@tapered@path\spath@to@add
  \edef\spath@to@add{\noexpand\pgfsyssoftpath@curvetosupportatoken{\spath@dx pt}{\spath@dy pt}}
  \ge@addto@macro\spath@tapered@path\spath@to@add
  \edef\spath@to@add{\noexpand\pgfsyssoftpath@curvetosupportbtoken{\spath@cx pt}{\spath@cy pt}}
  \ge@addto@macro\spath@tapered@path\spath@to@add
  \edef\spath@to@add{\noexpand\pgfsyssoftpath@curvetotoken{\spath@sx pt}{\spath@sy pt}}
  \ge@addto@macro\spath@tapered@path\spath@to@add
%    \end{macrocode}
% Close the path
%    \begin{macrocode}
  \g@addto@macro{\spath@tapered@path}{\pgfsyssoftpath@closepath}
}
%    \end{macrocode}
% \end{macro}
%
%
% \begin{macro}{\spath@last}
%    \begin{macrocode}
\def\spath@last#1{%
  \spath@real@length#1%
  \def\spath@last@path{}%
  \def\spath@rest@path{}%
  \def\spath@last@action##1##2##3{%
    \edef\spath@this@last@action{\string##1}
    \ifx\spath@this@last@action\spath@lineto
     \addtocounter{spath@length}{-1}%
     \ifnum\value{spath@length}=0\relax
      \let\spath@rest@path=\spath@last@path
      \edef\spath@last@path{\noexpand\pgfsyssoftpath@movetotoken{\spath@sx}{\spath@sy}}
     \fi
     \ifnum\value{spath@length}=1\relax
      \def\spath@sx{##2}
      \def\spath@sy{##3}
     \fi
    \else
    \ifx\spath@this@last@action\spath@curvetosupporta
     \addtocounter{spath@length}{-1}%
     \ifnum\value{spath@length}=0\relax
      \let\spath@rest@path=\spath@last@path
      \edef\spath@last@path{\noexpand\pgfsyssoftpath@movetotoken{\spath@sx}{\spath@sy}}
     \fi
    \else
    \ifx\spath@this@last@action\spath@curveto
     \ifnum\value{spath@length}=1\relax
      \def\spath@sx{##2}
      \def\spath@sy{##3}
     \fi
    \fi
    \fi
    \fi
    \def\spath@to@add{##1{##2}{##3}}
  }
  \let\spath@last@append=\ge@addto@macro
  \def\@temp{\spath@walk\spath@last@path\spath@last@action\spath@last@append}
  \expandafter\@temp#1\relax
}
%    \end{macrocode}
% \end{macro}
%
%
% \begin{macro}{\spath@taper}
%    \begin{macrocode}
\def\spath@taper#1{%
  \ifspath@taper@end
   \spath@last#1
   \spath@taper@in\spath@last@path
   \pgfsyssoftpath@setcurrentpath{\spath@tapered@path}
   \pgfsyssoftpath@flushcurrentpath
   \pgfusepath{fill}
  \else
   \let\spath@rest@path=#1
  \fi
  \ifspath@taper@start
   \spath@first\spath@rest@path
   \spath@taper@out\spath@start@path
   \pgfsyssoftpath@setcurrentpath{\spath@tapered@path}
   \pgfsyssoftpath@flushcurrentpath
   \pgfusepath{fill}
  \fi
  \pgfsyssoftpath@setcurrentpath{\spath@rest@path}
  \pgfsyssoftpath@flushcurrentpath
  \pgfusepath{stroke}
}
%    \end{macrocode}
% \end{macro}
%
%
% \begin{macro}{\spath@split}
%    \begin{macrocode}
\def\spath@split#1{%
  \let\spath@split@path=\pgfutil@empty
  \spath@real@length#1
  \ifnum\value{spath@length}=1\relax
%    \end{macrocode}
% Only one component, split in to three
%    \begin{macrocode}
   \expandafter\spath@find@taper@type#1\relax
   \ifx\spath@taper@type\spath@lineto
    \expandafter\spath@split@single@lineto#1\relax
   \else
   \ifx\spath@taper@type\spath@curvetosupporta
    \expandafter\spath@split@single@curveto#1\relax
   \fi
   \fi
  \else
  \ifnum\value{spath@length}=2\relax
%    \end{macrocode}
% Two components, split in to four
% Find first component type
%    \begin{macrocode}
   \expandafter\spath@find@taper@type#1\relax
   \ifx\spath@taper@type\spath@lineto
    \expandafter\spath@split@first@lineto#1\relax
   \else
   \ifx\spath@taper@type\spath@curvetosupporta
    \expandafter\spath@split@first@curveto#1\relax
   \fi
   \fi
   \expandafter\spath@find@taper@type\spath@split@tmp@path\relax
   \ifx\spath@taper@type\spath@lineto
    \expandafter\spath@split@second@lineto\spath@split@tmp@path\relax
   \else
   \ifx\spath@taper@type\spath@curvetosupporta
    \expandafter\spath@split@second@curveto\spath@split@tmp@path\relax
   \fi
   \fi
  \else
%    \end{macrocode}
% More than two, leave as is
%    \begin{macrocode}
   \let\spath@split@path=#1
  \fi
  \fi
}
%    \end{macrocode}
% \end{macro}
%
%
% \begin{macro}{\spath@single@split}
%    \begin{macrocode}
\def\spath@single@split{.3}
%    \end{macrocode}
% \end{macro}
%
%
% \begin{macro}{\spath@split@single@lineto}
%    \begin{macrocode}
\def\spath@split@single@lineto\pgfsyssoftpath@movetotoken#1#2\pgfsyssoftpath@linetotoken#3#4#5\relax{%
  \pgfmathsetmacro{\spath@sx}{(1-\spath@single@split)*#1 + \spath@single@split * #3}
  \pgfmathsetmacro{\spath@sy}{(1-\spath@single@split)*#2 + \spath@single@split * #4}
  \pgfmathsetmacro{\spath@ex}{(1-\spath@single@split)*#3 + \spath@single@split * #1}
  \pgfmathsetmacro{\spath@ey}{(1-\spath@single@split)*#4 + \spath@single@split * #2}
  \edef\spath@split@path{\noexpand\pgfsyssoftpath@movetotoken{#1}{#2}\noexpand\pgfsyssoftpath@linetotoken{\spath@sx     pt}{\spath@sy pt}\noexpand\pgfsyssoftpath@linetotoken{\spath@ex pt}{\spath@ey pt}\noexpand\pgfsyssoftpath@linetotoken{#3}{#4}}
}
%    \end{macrocode}
% \end{macro}
%
%
% \begin{macro}{\spath@split@single@curveto}
%    \begin{macrocode}
\def\spath@split@single@curveto\pgfsyssoftpath@movetotoken#1#2\pgfsyssoftpath@curvetosupportatoken#3#4\pgfsyssoftpath@curvetosupportbtoken#5#6\pgfsyssoftpath@curvetotoken#7#8#9\relax{%
  \pgfmathsetmacro{\spath@cx}{(1 - \spath@single@split)*#1 + \spath@single@split * #3}
  \pgfmathsetmacro{\spath@cy}{(1 - \spath@single@split)*#2 + \spath@single@split * #4}
  \pgfmathsetmacro{\spath@dx}{(1 - \spath@single@split)^2 * #1 + 2*\spath@single@split * (1 - \spath@single@split) * #3 + \spath@single@split^2 * #5}
  \pgfmathsetmacro{\spath@dy}{(1 - \spath@single@split)^2 * #2 + 2*\spath@single@split * (1 - \spath@single@split) * #4 + \spath@single@split^2 * #6}
  \pgfmathsetmacro{\spath@ex}{(1 - \spath@single@split)^3 * #1 + 3*\spath@single@split * (1 - \spath@single@split)^2 * #3 + 3*\spath@single@split^2 * (1 - \spath@single@split) * #5 + \spath@single@split^3 * #7}
  \pgfmathsetmacro{\spath@ey}{(1 - \spath@single@split)^3 * #2 + 3*\spath@single@split * (1 - \spath@single@split)^2 * #4 + 3*\spath@single@split^2 * (1 - \spath@single@split) * #6 + \spath@single@split^3 * #8}
  \edef\spath@split@path{\noexpand\pgfsyssoftpath@movetotoken{#1}{#2}\noexpand\pgfsyssoftpath@curvetosupportatoken{\spath@cx     pt}{\spath@cy pt}\noexpand\pgfsyssoftpath@curvetosupportbtoken{\spath@dx pt}{\spath@dy pt}\noexpand\pgfsyssoftpath@curvetotoken{\spath@ex pt}{\spath@ey pt}}
%    \end{macrocode}
% Should be some sort of optimisation to do here
%    \begin{macrocode}
  \pgfmathsetmacro{\spath@cx}{(1 - \spath@single@split)^2*\spath@single@split * #1 + (1 - 3 * \spath@single@split + 5 * \spath@single@split^2 - 3 * \spath@single@split^3) * #3 + (2 * \spath@single@split - 4 * \spath@single@split^2 + 3 * \spath@single@split^3) * #5 + \spath@single@split^2 * (1 - \spath@single@split) * #7}
  \pgfmathsetmacro{\spath@cy}{(1 - \spath@single@split)^2*\spath@single@split * #2 + (1 - 3 * \spath@single@split + 5 * \spath@single@split^2 - 3 * \spath@single@split^3) * #4 + (2 * \spath@single@split - 4 * \spath@single@split^2 + 3 * \spath@single@split^3) * #6 + \spath@single@split^2 * (1 - \spath@single@split) * #8}
  \pgfmathsetmacro{\spath@dx}{(1 - \spath@single@split)^2*\spath@single@split * #7 + (1 - 3 * \spath@single@split + 5 * \spath@single@split^2 - 3 * \spath@single@split^3) * #5 + (2 * \spath@single@split - 4 * \spath@single@split^2 + 3 * \spath@single@split^3) * #3 + \spath@single@split^2 * (1 - \spath@single@split) * #1}
  \pgfmathsetmacro{\spath@dy}{(1 - \spath@single@split)^2*\spath@single@split * #8 + (1 - 3 * \spath@single@split + 5 * \spath@single@split^2 - 3 * \spath@single@split^3) * #6 + (2 * \spath@single@split - 4 * \spath@single@split^2 + 3 * \spath@single@split^3) * #4 + \spath@single@split^2 * (1 - \spath@single@split) * #2}
  \pgfmathsetmacro{\spath@ex}{\spath@single@split^3 * #1 + 3 * \spath@single@split^2 * (1 - \spath@single@split) * #3 + 3 * \spath@single@split * (1 - \spath@single@split)^2 * #5 + (1 - \spath@single@split)^3 * #7}
  \pgfmathsetmacro{\spath@ey}{\spath@single@split^3 * #2 + 3 * \spath@single@split^2 * (1 - \spath@single@split) * #4 + 3 * \spath@single@split * (1 - \spath@single@split)^2 * #6 + (1 - \spath@single@split)^3 * #8}
  \edef\spath@to@add{\noexpand\pgfsyssoftpath@curvetosupportatoken{\spath@cx     pt}{\spath@cy pt}\noexpand\pgfsyssoftpath@curvetosupportbtoken{\spath@dx pt}{\spath@dy pt}\noexpand\pgfsyssoftpath@curvetotoken{\spath@ex pt}{\spath@ey pt}}
  \ge@addto@macro\spath@split@path\spath@to@add
  \pgfmathsetmacro{\spath@dx}{(1 - \spath@single@split)*#7 + \spath@single@split * #5}
  \pgfmathsetmacro{\spath@dy}{(1 - \spath@single@split)*#8 + \spath@single@split * #6}
  \pgfmathsetmacro{\spath@cx}{(1 - \spath@single@split)^2 * #7 + 2*\spath@single@split * (1 - \spath@single@split) * #5 + \spath@single@split^2 * #3}
  \pgfmathsetmacro{\spath@cy}{(1 - \spath@single@split)^2 * #8 + 2*\spath@single@split * (1 - \spath@single@split) * #6 + \spath@single@split^2 * #4}
  \edef\spath@to@add{\noexpand\pgfsyssoftpath@curvetosupportatoken{\spath@cx     pt}{\spath@cy pt}\noexpand\pgfsyssoftpath@curvetosupportbtoken{\spath@dx pt}{\spath@dy pt}\noexpand\pgfsyssoftpath@curvetotoken{#7}{#8}}
  \ge@addto@macro\spath@split@path\spath@to@add
}
%    \end{macrocode}
% \end{macro}
%
%
% \begin{macro}{\spath@double@split}
%    \begin{macrocode}
\def\spath@double@split{.5}
%    \end{macrocode}
% \end{macro}
%
%
% \begin{macro}{\spath@split@first@lineto}
%    \begin{macrocode}
\def\spath@split@first@lineto\pgfsyssoftpath@movetotoken#1#2\pgfsyssoftpath@linetotoken#3#4#5\relax{%
  \pgfmathsetmacro{\spath@sx}{(1-\spath@single@split)*#1 + \spath@single@split * #3}
  \pgfmathsetmacro{\spath@sy}{(1-\spath@single@split)*#2 + \spath@single@split * #4}
  \edef\spath@split@path{\noexpand\pgfsyssoftpath@movetotoken{#1}{#2}\noexpand\pgfsyssoftpath@linetotoken{\spath@sx     pt}{\spath@sy pt}\noexpand\pgfsyssoftpath@linetotoken{#3}{#4}}
  \edef\spath@split@tmp@path{\noexpand\pgfsyssoftpath@movetotoken{#3}{#4}}
  \g@addto@macro\spath@split@tmp@path{#5}
}
%    \end{macrocode}
% \end{macro}
%
%
% \begin{macro}{\spath@split@second@lineto}
%    \begin{macrocode}
\def\spath@split@second@lineto\pgfsyssoftpath@movetotoken#1#2\pgfsyssoftpath@linetotoken#3#4#5\relax{%
  \pgfmathsetmacro{\spath@sx}{(1-\spath@single@split)*#1 + \spath@single@split * #3}
  \pgfmathsetmacro{\spath@sy}{(1-\spath@single@split)*#2 + \spath@single@split * #4}
  \edef\spath@to@add{\noexpand\pgfsyssoftpath@linetotoken{\spath@sx pt}{\spath@sy pt}\noexpand\pgfsyssoftpath@linetotoken{#3}{#4}}
  \ge@addto@macro\spath@split@path\spath@to@add
}
%    \end{macrocode}
% \end{macro}
%
%
% \begin{macro}{\spath@split@first@curveto}
%    \begin{macrocode}
\def\spath@split@first@curveto\pgfsyssoftpath@movetotoken#1#2\pgfsyssoftpath@curvetosupportatoken#3#4\pgfsyssoftpath@curvetosupportbtoken#5#6\pgfsyssoftpath@curvetotoken#7#8#9\relax{%
  \pgfmathsetmacro{\spath@cx}{(1 - \spath@double@split)*#1 + \spath@double@split * #3}
  \pgfmathsetmacro{\spath@cy}{(1 - \spath@double@split)*#2 + \spath@double@split * #4}
  \pgfmathsetmacro{\spath@dx}{(1 - \spath@double@split)^2 * #1 + 2*\spath@double@split * (1 - \spath@double@split) * #3 + \spath@double@split^2 * #5}
  \pgfmathsetmacro{\spath@dy}{(1 - \spath@double@split)^2 * #2 + 2*\spath@double@split * (1 - \spath@double@split) * #4 + \spath@double@split^2 * #6}
  \pgfmathsetmacro{\spath@ex}{(1 - \spath@double@split)^3 * #1 + 3*\spath@double@split * (1 - \spath@double@split)^2 * #3 + 3*\spath@double@split^2 * (1 - \spath@double@split) * #5 + \spath@double@split^3 * #7}
  \pgfmathsetmacro{\spath@ey}{(1 - \spath@double@split)^3 * #2 + 3*\spath@double@split * (1 - \spath@double@split)^2 * #4 + 3*\spath@double@split^2 * (1 - \spath@double@split) * #6 + \spath@double@split^3 * #8}
  \edef\spath@split@path{\noexpand\pgfsyssoftpath@movetotoken{#1}{#2}\noexpand\pgfsyssoftpath@curvetosupportatoken{\spath@cx     pt}{\spath@cy pt}\noexpand\pgfsyssoftpath@curvetosupportbtoken{\spath@dx pt}{\spath@dy pt}\noexpand\pgfsyssoftpath@curvetotoken{\spath@ex pt}{\spath@ey pt}}
  \pgfmathsetmacro{\spath@dx}{(1 - \spath@double@split)*#5 + \spath@double@split * #7}
  \pgfmathsetmacro{\spath@dy}{(1 - \spath@double@split)*#6 + \spath@double@split * #8}
  \pgfmathsetmacro{\spath@cx}{(1 - \spath@double@split)^2 * #3 + 2*\spath@double@split * (1 - \spath@double@split) * #5 + \spath@double@split^2 * #7}
  \pgfmathsetmacro{\spath@cy}{(1 - \spath@double@split)^2 * #4 + 2*\spath@double@split * (1 - \spath@double@split) * #6 + \spath@double@split^2 * #8}
  
  \edef\spath@to@add{\noexpand\pgfsyssoftpath@curvetosupportatoken{\spath@cx     pt}{\spath@cy pt}\noexpand\pgfsyssoftpath@curvetosupportbtoken{\spath@dx pt}{\spath@dy pt}\noexpand\pgfsyssoftpath@curvetotoken{#7}{#8}}
  \ge@addto@macro\spath@split@path\spath@to@add
  \edef\spath@split@tmp@path{\noexpand\pgfsyssoftpath@movetotoken{#7}{#8}}
  \g@addto@macro\spath@split@tmp@path{#9}
}
%    \end{macrocode}
% \end{macro}
%
%
% \begin{macro}{\spath@split@second@curveto}
%    \begin{macrocode}
\def\spath@split@second@curveto\pgfsyssoftpath@movetotoken#1#2\pgfsyssoftpath@curvetosupportatoken#3#4\pgfsyssoftpath@curvetosupportbtoken#5#6\pgfsyssoftpath@curvetotoken#7#8#9\relax{%
  \pgfmathsetmacro{\spath@cx}{(1 - \spath@double@split)*#1 + \spath@double@split * #3}
  \pgfmathsetmacro{\spath@cy}{(1 - \spath@double@split)*#2 + \spath@double@split * #4}
  \pgfmathsetmacro{\spath@dx}{(1 - \spath@double@split)^2 * #1 + 2*\spath@double@split * (1 - \spath@double@split) * #3 + \spath@double@split^2 * #5}
  \pgfmathsetmacro{\spath@dy}{(1 - \spath@double@split)^2 * #2 + 2*\spath@double@split * (1 - \spath@double@split) * #4 + \spath@double@split^2 * #6}
  \pgfmathsetmacro{\spath@ex}{(1 - \spath@double@split)^3 * #1 + 3*\spath@double@split * (1 - \spath@double@split)^2 * #3 + 3*\spath@double@split^2 * (1 - \spath@double@split) * #5 + \spath@double@split^3 * #7}
  \pgfmathsetmacro{\spath@ey}{(1 - \spath@double@split)^3 * #2 + 3*\spath@double@split * (1 - \spath@double@split)^2 * #4 + 3*\spath@double@split^2 * (1 - \spath@double@split) * #6 + \spath@double@split^3 * #8}
  \edef\spath@to@add{\noexpand\pgfsyssoftpath@movetotoken{#1}{#2}\noexpand\pgfsyssoftpath@curvetosupportatoken{\spath@cx     pt}{\spath@cy pt}\noexpand\pgfsyssoftpath@curvetosupportbtoken{\spath@dx pt}{\spath@dy pt}\noexpand\pgfsyssoftpath@curvetotoken{\spath@ex pt}{\spath@ey pt}}
  \ge@addto@macro\spath@split@path\spath@to@add
  \pgfmathsetmacro{\spath@dx}{(1 - \spath@double@split)*#5 + \spath@double@split * #7}
  \pgfmathsetmacro{\spath@dy}{(1 - \spath@double@split)*#6 + \spath@double@split * #8}
  \pgfmathsetmacro{\spath@cx}{(1 - \spath@double@split)^2 * #3 + 2*\spath@double@split * (1 - \spath@double@split) * #5 + \spath@double@split^2 * #7}
  \pgfmathsetmacro{\spath@cy}{(1 - \spath@double@split)^2 * #4 + 2*\spath@double@split * (1 - \spath@double@split) * #6 + \spath@double@split^2 * #8}
  
  \edef\spath@to@add{\noexpand\pgfsyssoftpath@curvetosupportatoken{\spath@cx     pt}{\spath@cy pt}\noexpand\pgfsyssoftpath@curvetosupportbtoken{\spath@dx pt}{\spath@dy pt}\noexpand\pgfsyssoftpath@curvetotoken{#7}{#8}}
  \ge@addto@macro\spath@split@path\spath@to@add
}
%    \end{macrocode}
% \end{macro}
%
%    \begin{macrocode}
\endinput%    \end{macrocode}
%
% \iffalse
%</package>
% \fi
%
% \Finale
\endinput
