\immediate\write18{tex spath.dtx}
\documentclass{ltxdoc}
\usepackage[T1]{fontenc}
\usepackage{lmodern}
\usepackage{morefloats}
\usepackage{tikz}
\usepackage{spath}
\usepackage[numbered]{hypdoc}
\definecolor{lstbgcolor}{rgb}{0.9,0.9,0.9} 
 
\usepackage{listings}
\lstloadlanguages{[LaTeX]TeX}
\lstset{breakatwhitespace=true,breaklines=true,language=TeX}
 
\usepackage{fancyvrb}

\newenvironment{example}
  {\VerbatimEnvironment
   \begin{VerbatimOut}{example.out}}
  {\end{VerbatimOut}
   \begin{center}
   \setlength{\parindent}{0pt}
   \fbox{\begin{minipage}{.9\linewidth}
     \lstset{breakatwhitespace=true,breaklines=true,language=TeX,basicstyle=\small}
     \lstinputlisting[]{example.out}
   \end{minipage}}

   \fbox{\begin{minipage}{.9\linewidth}
     \input{example.out}
   \end{minipage}}
\end{center}
}

\providecommand*{\url}{\texttt}
\GetFileInfo{spath.sty}

\title{The \textsf{spath} Package: Documentation}
\author{Andrew Stacey \\ \url{stacey@math.ntnu.no}}
\date{\fileversion~from \filedate}

\begin{document}

\maketitle

\section{Introduction}

The \texttt{spath} library is an object-oriented interface to the \emph{soft path} system of PGF.
It allows one to save, manipulate, and use \emph{soft paths} in a more general way than is currently easy with TikZ/PGF.
This library was developed for use with the \texttt{calligraphy} package since that required various path manipulations that would have been hard to do otherwise.

There are several steps in the TikZ/PGF system between writing \Verb+\draw (0,0) -- (1,0);+ and a line appearing in the document.
The command just given is a TikZ command.
The TikZ frontend translates that into PGF commands.
These are then processed into a \emph{soft path}.
The soft path can be manipulated further before being ``baked'' into a hard path which is then written out into the output file via an appropriate driver.

A soft path is a simple object.
It consists of a list of tokens of the form \Verb+\pgfsyssoftpath@<something>token{x pt}{y pt}+\footnote{Except for the \Verb+closepath+ token which omits the final \Verb+token+.}.
There are only a handful of possible \Verb+<something>+s.
It is therefore quite straightforward to manipulate these paths in an orderly fashion.
However, the PGF code does not provide a particular method for doing complicated manipulations on these paths.
This library introduces some such methods.

One thing to note at the outset is that this library makes no attempt to optimise the code and so if the soft path is very long, it will take a considerable time to do any of these manipulations.
The PGF soft path system does have some optimisations to handle this.

\section{Usage}

Using this library is relatively straightforward.
After loading TikZ/PGF, simply \Verb+\usepackage{spath}+ in the preamble.
A typical example of using the library is in the following example.

\begin{example}
\begin{tikzpicture}
\path[save path=\tmppath] (-1,0) -- (0,0) .. controls +(1,0) and +(1,0) .. (1,1);
\pgfoonew \mypath=new spath(\tmppath)
\mypath.prepare()
\mypath.reverse path()
\mypath.use path with tikz(draw)
\mypath.translate path(,0cm,-2cm)
\mypath.at least three()
\mypath.split path by real length(\fpath,\mpath,1)
\begin{scope}[ultra thick]
\fpath.taper out()
\fpath.use path with tikz(fill)
\mpath.split path by real length(\mmpath,\epath,-1)
\epath.reverse path()
\epath.taper out()
\epath.use path with tikz(fill)
\mmpath.use path with tikz(draw)
\end{scope}
\end{tikzpicture}
\end{example}

In this example, we start by defining a path in the usual TikZian manner.
However, instead of doing something with the path, we save it.
This saves the raw path as a macro.
In order to use this library on that path, we have to convert it into an \Verb+spath+ object.
This is done with the next line, starting \Verb+\pgfoonew+ (note the spacing: there must be no space after the \Verb+=+).
This command declares \Verb+\mypath+ to be an instance of the \Verb+spath+ class and initialises it with the path stored in \Verb+\tmppath+.
(For scoping reasons, we can't define a PGF key to be given to the original \Verb+\path+ command to do this in one go.)
This can now be manipulated with the \Verb+spath+ methods.
We shall explain the actual methods later so shall not go into detail now.

\section{Attributes and Methods}

The \texttt{spath} library defines two classes: the \Verb+spath+ and \Verb+spath component+.
The \Verb+spath+ is a soft path, the \Verb+spath component+ is an array of \Verb+spath+ objects.

\subsection{The \texttt{spath} Object}

The following is a list of the attributes that an \Verb+spath+ object has.
The class operates on a lazy system: attributes are not automatically worked out so they will not have values initially.
However, as soon as you do something that needs one to have a value it will be computed.

\begin{itemize}
\item \Verb+path+

This holds the actual soft path.

\item \Verb+length+

This (when initialised) is the length of the soft path in number of soft path tokens.

\item \Verb+real length+

This is the number of actual drawing tokens in the soft path; that is, the number of \Verb+lineto+ and \Verb+curveto+ tokens.

(This ought to also count \Verb+rect+, but I don't know of a TikZ command that actually produces a \Verb+rect+.)

\item \Verb+number of components+

This is the number of components of the path when divided up by \Verb+moveto+s.
It is actually the number of \Verb+moveto+s so successive \Verb+moveto+s will each count.

\item \Verb+initial point+

This is the coordinates of the start of the path.
It is of the form \Verb+\pgfpoint{x coordinate}{y coordinate}+ so when executed will set \Verb+\pgf@x+ and \Verb+\pgf@y+ to the corresponding coordinates.

\item \Verb+final point+

This is the coordinates of the end of the path.

\item \Verb+first action+

This is the first drawing action on the path (that is, not the initial \Verb+moveto+).

\item \Verb+last action+

This is the last action on the path (which might be a \Verb+moveto+).

\item \Verb+prepared+

The attributes are not calculated initially, but are calculated as needed.
There is a method available for calculating them all in one fell swoop.
If this has been called, this attribute is set so that the path knows its attributes have all be computed and don't need to be computed again.

\item \Verb+taper line width+

When tapering a path, it is necessary to set both the current line width and a width to taper to.
This holds the latter.

\end{itemize}

The following is a list of the methods available for an \Verb+spath+ object with a brief explanation of what each does.
The contents of the parentheses are the arguments for the method.

\begin{itemize}
\item \Verb+spath(#1)+

This is the creator method, it is called automatically when the object is created.
The argument, if given, is a macro containing a soft path which is used to set the \Verb+path+ attribute.

\item \Verb+value(#1)+

This method inserts the value of the attribute passed to it into the token stream.

\item \Verb+set(#1,#2)+

This method sets the attribute passed to it as the first argument to the second argument.

\item \Verb+let(#1,#2)+

This method lets the attribute passed to it as the first argument to be the second argument.
The second argument ought to be a macro, therefore.

\item \Verb+get(#1,#2)+

This method assigns the second argument to the value of the attribute passed to it as the first argument.
The second argument ought to be a macro, therefore.

\item \Verb+show(#1)+

This method shows the value of the attribute in the logs.
It uses \Verb+\show+ internally.

\item \Verb+clone(#1)+

This method clones the \Verb+spath+ object into the macro passed as the argument. 

\item \Verb+translate path(#1,#2,#3)+

This method translates the path by \Verb+#2+ in the x-direction and \Verb+#3+ in the y-direction.
If \Verb+#1+ is not empty, it should be a macro which is used to store the translated object (it will be an \Verb+spath+ object).
If \Verb+#1+ is empty, the current object is modified.

\item \Verb+length()+

This method inserts the length of the path in the token stream, computing it if it is not already known.
(It is this computation which makes it differ from the \Verb+value+ method.)

\item \Verb+real length()+

This method inserts the real length of the path in the token stream, computing it if it is not already known.

\item \Verb+number of components()+

This method inserts the number of components of the path in the token stream, computing it if it is not already known.

\item \Verb+initial point()+

This method inserts the number of components of the path in the token stream, computing it if it is not already known.

\item \Verb+final point()+

This method inserts the number of components of the path in the token stream, computing it if it is not already known.

\item \Verb+reverse path(#1)+

This method reverses the soft path.
If the argument is given, the reverse path is stored in it as a new \Verb+spath+ object.
If not, the current path is modified.

\item \Verb+prepare()+
\item \Verb+at least three()+
\item \Verb+taper out()+
\item \Verb+split path by length(#1,#2,#3)+
\item \Verb+split path by real length(#1,#2,#3)+
\item \Verb+split path by component(#1,#2,#3)+
\item \Verb+split(#1,#2,#3)+
\item \Verb+reprocess path()+
\item \Verb+set as current path()+
\item \Verb+get from current path()+
\item \Verb+use path(#1)+
\item \Verb+use path with tikz(#1)+
\item \Verb+concatenate(#1,#2)+
\item \Verb+concatenate with lineto(#1,#2)+
\item \Verb+weld(#1,#2)+
\item \Verb+close()+
\end{itemize}


\begin{itemize}
\item \Verb+path+
\item \Verb+next component+
\item \Verb+previous component+
\end{itemize}

\begin{itemize}
\item \Verb+spath component(#1)+
\item \Verb+value(#1)+
\item \Verb+set(#1,#2)+
\item \Verb+let(#1,#2)+
\item \Verb+get(#1,#2)+
\item \Verb+show(#1)+
\item \Verb+set path(#1)+
\item \Verb+apply to paths(#1,#2)+
\item \Verb+apply to previous paths(#1,#2)+
\end{itemize}

\end{document}