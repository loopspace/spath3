% \iffalse meta-comment
%<*internal>
\iffalse
%</internal>
%<*readme>
----------------------------------------------------------------
spath --- Functions for manipulating PGF soft paths
E-mail: stacey@math.ntnu.no
Released under the LaTeX Project Public License v1.3c or later
See http://www.latex-project.org/lppl.txt
----------------------------------------------------------------

This package defines some functions used to manipulate PGFs soft paths.
%</readme>
%<*internal>
\fi
\def\nameofplainTeX{plain}
\ifx\fmtname\nameofplainTeX\else
  \expandafter\begingroup
\fi
%</internal>
%<*install>
\input docstrip.tex
\keepsilent
\askforoverwritefalse
\preamble
----------------------------------------------------------------
spath --- Functions for manipulating PGF soft paths
E-mail: stacey@math.ntnu.no
Released under the LaTeX Project Public License v1.3c or later
See http://www.latex-project.org/lppl.txt
----------------------------------------------------------------

\endpreamble
\postamble

Copyright (C) 2011 by Andrew Stacey <stacey@math.ntnu.no>

This work may be distributed and/or modified under the
conditions of the LaTeX Project Public License (LPPL), either
version 1.3c of this license or (at your option) any later
version.  The latest version of this license is in the file:

http://www.latex-project.org/lppl.txt

This work is "maintained" (as per LPPL maintenance status) by
Andrew Stacey.

This work consists of the file  spath.dtx
and the derived files           spath.ins,
                                spath.pdf, and
                                spath.sty.

\endpostamble
\usedir{tex/latex/spath}
\generate{
  \file{\jobname.sty}{\from{\jobname.dtx}{package}}
}
%</install>
%<install>\endbatchfile
%<*internal>
\usedir{source/latex/spath}
\generate{
  \file{\jobname.ins}{\from{\jobname.dtx}{install}}
}
\nopreamble\nopostamble
\usedir{doc/latex/demopkg}
\generate{
  \file{README.txt}{\from{\jobname.dtx}{readme}}
}
\ifx\fmtname\nameofplainTeX
  \expandafter\endbatchfile
\else
  \expandafter\endgroup
\fi
%</internal>
%<*package>
\NeedsTeXFormat{LaTeX2e}
\ProvidesPackage{spath}[2011/06/03 v1.0 Functions for manipulating PGF soft paths]
%</package>
%<*driver>
\documentclass{ltxdoc}
\usepackage[T1]{fontenc}
\usepackage{lmodern}
%\usepackage{morefloats}
\usepackage{tikz}
\usepackage{\jobname}
\usepackage[numbered]{hypdoc}
\definecolor{lstbgcolor}{rgb}{0.9,0.9,0.9} 
 
\usepackage{listings}
\lstloadlanguages{[LaTeX]TeX}
\lstset{breakatwhitespace=true,breaklines=true,language=TeX}
 
\usepackage{fancyvrb}

\newenvironment{example}
  {\VerbatimEnvironment
   \begin{VerbatimOut}[gobble=2]{example.out}}
  {\end{VerbatimOut}
   \begin{center}
%   \setlength{\parindent}{0pt}
   \fbox{\begin{minipage}{.9\linewidth}
     \lstset{breakatwhitespace=true,breaklines=true,language=TeX,basicstyle=\small}
     \lstinputlisting[]{example.out}
   \end{minipage}}

   \fbox{\begin{minipage}{.9\linewidth}
     \input{example.out}
   \end{minipage}}
\end{center}
}
\EnableCrossrefs
\CodelineIndex
\RecordChanges
\begin{document}
  \DocInput{\jobname.dtx}
\end{document}
%</driver>
% \fi
%
%
% \CharacterTable
%  {Upper-case    \A\B\C\D\E\F\G\H\I\J\K\L\M\N\O\P\Q\R\S\T\U\V\W\X\Y\Z
%   Lower-case    \a\b\c\d\e\f\g\h\i\j\k\l\m\n\o\p\q\r\s\t\u\v\w\x\y\z
%   Digits        \0\1\2\3\4\5\6\7\8\9
%   Exclamation   \!     Double quote  \"     Hash (number) \#
%   Dollar        \$     Percent       \%     Ampersand     \&
%   Acute accent  \'     Left paren    \(     Right paren   \)
%   Asterisk      \*     Plus          \+     Comma         \,
%   Minus         \-     Point         \.     Solidus       \/
%   Colon         \:     Semicolon     \;     Less than     \<
%   Equals        \=     Greater than  \>     Question mark \?
%   Commercial at \@     Left bracket  \[     Backslash     \\
%   Right bracket \]     Circumflex    \^     Underscore    \_
%   Grave accent  \`     Left brace    \{     Vertical bar  \|
%   Right brace   \}     Tilde         \~}
%
%
% \changes{1.0}{2011/05/03}{Converted to DTX file}
%
% \DoNotIndex{\newcommand,\newenvironment}
%
% \providecommand*{\url}{\texttt}
% \GetFileInfo{spath.dtx}
% \title{The \textsf{spath} package}
% \author{Andrew Stacey \\ \url{stacey@math.ntnu.no}}
% \date{\fileversion~from \filedate}
%
%
% \maketitle
%
% 
% \section{Introduction}
%
% \StopEventually{}
%
% \section{Implementation}
%
% \iffalse
%<*package>
% \fi
%
% Set up tokens for comparision
%    \begin{macrocode}
\long\def\ge@addto@macro#1#2{%
  \begingroup
  \toks@\expandafter\expandafter\expandafter{\expandafter#1#2}%
  \xdef#1{\the\toks@}%
  \endgroup}

\long\def\ge@addbefore@macro#1#2{%
  \begingroup
  \toks@\expandafter\expandafter\expandafter{\expandafter#2#1}%
  \xdef#1{\the\toks@}%
  \endgroup}

\long\def\g@addbefore@macro#1#2{%
  \def\@temp{#2}
  \ge@addbefore@macro{#1}{\@temp}}

\def\spath@define@transfer#1\relax{%
  \expandafter\gdef\csname spath@tr@#1\endcsname##1##2{%
    \pgf@xa=##1\relax
    \pgf@ya=##2\relax
    \advance\pgf@xa by \spath@trx
    \advance\pgf@ya by \spath@try
    \edef\spath@tmp{\expandafter\noexpand\csname pgfsyssoftpath@#1token\endcsname{\the\pgf@xa}{\the\pgf@ya}}
    \ge@addto@macro\spath@tmppath\spath@tmp
  }
}

\def\spath@define@length#1\relax{%
  \expandafter\gdef\csname spath@len@#1\endcsname##1##2{%
    \stepcounter{spath@length}
  }
}

\def\spath@define@components#1\relax{%
  \expandafter\gdef\csname spath@comp@#1\endcsname##1##2{%
  }
}

\def\spath@define@reallength#1\relax{%
  \expandafter\gdef\csname spath@rlen@#1\endcsname##1##2{%
  }
}

\def\spath@define@reverse#1\relax{%
  \expandafter\gdef\csname spath@rev@#1\endcsname##1##2{%
    \edef\spath@tmp{{##1}{##2}\expandafter\noexpand\csname pgfsyssoftpath@#1token\endcsname}
    \ge@addbefore@macro\spath@tmppath\spath@tmp
  }
}

\def\spath@define@start#1\relax{%
  \expandafter\gdef\csname spath@start@#1\endcsname##1##2{%
    \spath@sx=##1\relax
    \spath@sy=##2\relax
    \spath@start@reinit
  }
}

\def\spath@define@end#1\relax{%
  \expandafter\gdef\csname spath@end@#1\endcsname##1##2{%
    \spath@ex=##1\relax
    \spath@ey=##2\relax
  }
}

\def\spath@define@array#1\relax{%
  \expandafter\gdef\csname spath@array@#1\endcsname##1##2{%
    \edef\spath@tmp{\expandafter\noexpand\csname       pgfsyssoftpath@#1token\endcsname{##1}{##2}}
    \ge@addto@macro\spath@tmppath\spath@tmp
  }
}

\def\spath@define@prepare#1\relax{%
  \expandafter\gdef\csname spath@prepare@#1\endcsname##1##2{%
    \edef\spath@tmp{\expandafter\noexpand\csname       pgfsyssoftpath@#1token\endcsname{##1}{##2}}
    \ge@addto@macro\spath@tmppath\spath@tmp
  }
}

\let\spath@translate@init=\pgfutil@empty
\let\spath@length@init=\pgfutil@empty
\let\spath@components@init=\pgfutil@empty
\let\spath@reallength@init=\pgfutil@empty
\let\spath@reverse@init=\pgfutil@empty
\let\spath@start@init=\pgfutil@empty
\let\spath@start@reinit=\pgfutil@empty
\let\spath@end@init=\pgfutil@empty
\let\spath@array@init=\pgfutil@empty
\let\spath@prepare@init=\pgfutil@empty

\def\spath@closepath{closepath}

\foreach \spath@cpt in {
  moveto,
  lineto,
  curvetosupporta,
  curvetosupportb,
  curveto,
  rectcorner,
  rectsize,
  closepath%
} {
  \ifx\spath@cpt\spath@closepath
  \let\spath@cptfull\spath@cpt
  \else
  \edef\spath@cptfull{\spath@cpt token}
  \fi
% Save token names for comparision
\expandafter\xdef\csname spath@\spath@cpt\endcsname{\expandafter\string\csname pgfsyssoftpath@\spath@cptfull\endcsname}

% Translation
\expandafter\spath@define@transfer\spath@cpt\relax

\edef\spath@tmp{\noexpand\let\expandafter\noexpand\csname pgfsyssoftpath@\spath@cptfull\endcsname=\expandafter\noexpand\csname spath@tr@\spath@cpt \endcsname}

\ge@addto@macro\spath@translate@init\spath@tmp

% Counting total length
\expandafter\spath@define@length\spath@cpt\relax

\edef\spath@tmp{\noexpand\let\expandafter\noexpand\csname pgfsyssoftpath@\spath@cptfull\endcsname=\expandafter\noexpand\csname spath@len@\spath@cpt \endcsname}

\ge@addto@macro\spath@length@init\spath@tmp

% Counting components
\expandafter\spath@define@components\spath@cpt\relax

\edef\spath@tmp{\noexpand\let\expandafter\noexpand\csname pgfsyssoftpath@\spath@cptfull\endcsname=\expandafter\noexpand\csname spath@comp@\spath@cpt \endcsname}

\ge@addto@macro\spath@components@init\spath@tmp

% Counting real length
\expandafter\spath@define@reallength\spath@cpt\relax

\edef\spath@tmp{\noexpand\let\expandafter\noexpand\csname pgfsyssoftpath@\spath@cptfull\endcsname=\expandafter\noexpand\csname spath@rlen@\spath@cpt \endcsname}

\ge@addto@macro\spath@reallength@init\spath@tmp

% Reversal
\expandafter\spath@define@reverse\spath@cpt\relax

\edef\spath@tmp{\noexpand\let\expandafter\noexpand\csname pgfsyssoftpath@\spath@cptfull\endcsname=\expandafter\noexpand\csname spath@rev@\spath@cpt \endcsname}

\ge@addto@macro\spath@reverse@init\spath@tmp

% Initial coordinates
\expandafter\spath@define@start\spath@cpt\relax

\edef\spath@tmp{\noexpand\let\expandafter\noexpand\csname pgfsyssoftpath@\spath@cptfull\endcsname=\expandafter\noexpand\csname spath@start@\spath@cpt \endcsname}

\ge@addto@macro\spath@start@init\spath@tmp

\edef\spath@tmp{\noexpand\let\expandafter\noexpand\csname pgfsyssoftpath@\spath@cptfull\endcsname=\noexpand\@gobbletwo}

\ge@addto@macro\spath@start@reinit\spath@tmp

% Final coordinates
\expandafter\spath@define@end\spath@cpt\relax

\edef\spath@tmp{\noexpand\let\expandafter\noexpand\csname pgfsyssoftpath@\spath@cptfull\endcsname=\expandafter\noexpand\csname spath@end@\spath@cpt \endcsname}

\ge@addto@macro\spath@end@init\spath@tmp

% Split in to array
\expandafter\spath@define@array\spath@cpt\relax

\edef\spath@tmp{\noexpand\let\expandafter\noexpand\csname pgfsyssoftpath@\spath@cptfull\endcsname=\expandafter\noexpand\csname spath@array@\spath@cpt \endcsname}

\ge@addto@macro\spath@array@init\spath@tmp

% Prepare a path: split it, reverse the components, figure out endpoints
% Actions are too complicated to specify a template so just create initialisation code

\edef\spath@tmp{\noexpand\let\expandafter\noexpand\csname pgfsyssoftpath@\spath@cptfull\endcsname=\expandafter\noexpand\csname spath@prepare@\spath@cpt \endcsname}

\ge@addto@macro\spath@prepare@init\spath@tmp
}

% Correction for counting components: count movetos
\def\spath@comp@moveto#1#2{%
  \stepcounter{spath@length}
}

% Correction for counting real length: count linetos and curvetos
\def\spath@rlen@lineto#1#2{%
  \stepcounter{spath@length}
}
\def\spath@rlen@curveto#1#2{%
  \stepcounter{spath@length}
}

% Correction for reversing curvetos
\def\spath@rev@curvetosupporta#1#2{%
  \edef\spath@tmp{{#1}{#2}\noexpand\pgfsyssoftpath@curvetotoken}
  \ge@addbefore@macro\spath@tmppath\spath@tmp
}
\def\spath@rev@curvetosupportb#1#2{%
  \edef\spath@tmp{{#1}{#2}\noexpand\pgfsyssoftpath@curvetosupportbtoken}
  \ge@addbefore@macro\spath@tmppath\spath@tmp
}
\def\spath@rev@curveto#1#2{%
  \edef\spath@tmp{{#1}{#2}\noexpand\pgfsyssoftpath@curvetosupportatoken}
  \ge@addbefore@macro\spath@tmppath\spath@tmp
}

% Correction for reversing closepaths
\def\spath@rev@closepath#1#2{%
  \edef\spath@tmp{\noexpand\pgfsyssoftpath@closepath{#1}{#2}}
  \ge@addto@macro\spath@tmppath\spath@tmp
}

% Correction for splitting in to an array
\def\spath@array@moveto#1#2{%
  \ifx\spath@tmppath\pgfutil@empty
  \else
   \expandafter\global\expandafter\let\csname spath@array@\spath@path@name @\the\value{spath@array}\endcsname=\spath@tmppath
   \def\spath@tmppath{\pgfsyssoftpath@movetotoken{#1}{#2}}
  \fi
  \stepcounter{spath@array}
}

% Preparing a path: splitting, reversing, endpoints
\def\spath@prepare@moveto#1#2{%
  \ifx\spath@tmppath\pgfutil@empty
  \else
   \expandafter\global\expandafter\let\csname spath@array@\spath@path@name @\the\value{spath@array}\endcsname=\spath@tmppath
   \g@addbefore@macro\spath@tmppathrev{\pgfsyssoftpath@movetotoken}
   \expandafter\global\expandafter\let\csname spath@array@\spath@path@name @\the\value{spath@array}@rev\endcsname=\spath@tmppathrev
   \expandafter\xdef\csname spath@array@\spath@path@name @\the\value{spath@array}@end\endcsname{\noexpand\pgfqpoint{\the\spath@ex}{\the\spath@ey}}
   \expandafter\xdef\csname spath@array@\spath@path@name @\the\value{spath@array}@length\endcsname{\the\value{spath@length}}
   \expandafter\global\expandafter\let\csname spath@array@\spath@path@name @\the\value{spath@array}@first\endcsname=\spath@first
   \expandafter\global\expandafter\let\csname spath@array@\spath@path@name @\the\value{spath@array}@last\endcsname=\spath@last
  \fi
   \def\spath@tmppath{\pgfsyssoftpath@movetotoken{#1}{#2}}
   \def\spath@tmppathrev{{#1}{#2}}
   \let\spath@first=\pgfutil@empty
   \let\spath@last=\pgfutil@empty
  \stepcounter{spath@array}
  \expandafter\xdef\csname spath@array@\spath@path@name @\the\value{spath@array}@start\endcsname{\noexpand\pgfqpoint{#1}{#2}}
  \spath@ex=#1\relax
  \spath@ey=#2\relax
  \setcounter{spath@length}{0}
}

\def\spath@prepare@lineto#1#2{%
  \g@addto@macro\spath@tmppath{\pgfsyssoftpath@linetotoken{#1}{#2}}
  \g@addbefore@macro\spath@tmppathrev{{#1}{#2}\pgfsyssoftpath@linetotoken}
  \spath@ex=#1\relax
  \spath@ey=#2\relax
  \stepcounter{spath@length}
  \def\spath@last{lineto}
  \ifx\spath@first\pgfutil@empty
  \def\spath@first{lineto}
  \fi
}

\def\spath@prepare@curveto#1#2{%
  \g@addto@macro\spath@tmppath{\pgfsyssoftpath@curvetotoken{#1}{#2}}
  \g@addbefore@macro\spath@tmppathrev{{#1}{#2}\pgfsyssoftpath@curvetosupportatoken}
  \spath@ex=#1\relax
  \spath@ey=#2\relax
  \stepcounter{spath@length}
  \def\spath@last{curveto}
  \ifx\spath@first\pgfutil@empty
  \def\spath@first{curveto}
  \fi
}

\def\spath@prepare@curvetosupporta#1#2{%
  \g@addto@macro\spath@tmppath{\pgfsyssoftpath@curvetosupportatoken{#1}{#2}}
  \g@addbefore@macro\spath@tmppathrev{{#1}{#2}\pgfsyssoftpath@curvetotoken}
  \spath@ex=#1\relax
  \spath@ey=#2\relax
}

\def\spath@prepare@curvetosupportb#1#2{%
  \g@addto@macro\spath@tmppath{\pgfsyssoftpath@curvetosupportbtoken{#1}{#2}}
  \g@addbefore@macro\spath@tmppathrev{{#1}{#2}\pgfsyssoftpath@curvetosupportbtoken}
  \spath@ex=#1\relax
  \spath@ey=#2\relax
}

% Also need to handle the closepath variants

\newdimen\spath@trx
\newdimen\spath@try
\newdimen\spath@sx
\newdimen\spath@sy
\newdimen\spath@ex
\newdimen\spath@ey
\newcounter{spath@length}
\newcounter{spath@array}

\def\spath@translate#1#2#3{%
  \let\spath@tmppath=\pgfutil@empty
  \begingroup
  \spath@trx=#2\relax
  \spath@try=#3\relax
  \spath@translate@init
  #1
  \endgroup
}

\def\spath@length#1{%
  \setcounter{spath@length}{0}
  \begingroup
  \spath@length@init
  #1
  \endgroup
}

\def\spath@components#1{%
  \setcounter{spath@length}{0}
  \begingroup
  \spath@components@init
  #1
  \endgroup
}

\def\spath@reallength#1{%
  \setcounter{spath@length}{0}
  \begingroup
  \spath@reallength@init
  #1
  \endgroup
}

\def\spath@reverse#1{%
  \let\spath@tmppath=\pgfutil@empty
  \begingroup
  \spath@reverse@init
  #1
  \g@addbefore@macro\spath@tmppath\pgfsyssoftpath@movetotoken
  \expandafter\spath@trimlast\spath@tmppath\relax
  \endgroup
}

\def\spath@trimlast#1\pgfsyssoftpath@movetotoken\relax{\gdef\spath@tmppath{#1}}

\def\spath@start#1{%
  \begingroup
  \spath@start@init
  #1
  \global\spath@sx=\spath@sx
  \global\spath@sy=\spath@sy
  \endgroup
}

\def\spath@end#1{%
  \begingroup
  \spath@end@init
  #1
  \global\spath@ex=\spath@ex
  \global\spath@ey=\spath@ey
  \endgroup
}

\def\spath@cat#1#2{
  \let\spath@tmppath=#1
  \expandafter\spath@trimfirst#2\relax
  \ge@addto@macro\spath@tmppath\spath@trimmed@path
}

\def\spath@trimfirst#1#2#3#4\relax{%
  \edef\this@action{\string#1}%
  \ifx\this@action\spath@moveto
  \def\spath@trimmed@path{#4}%
  \else
  \def\spath@trimmed@path{#1{#2}{#3}#4}%
  \fi
}

\def\spath@array#1#2{
  \let\spath@tmppath=\pgfutil@empty
  \begingroup
  \setcounter{spath@array}{0}
  \def\spath@path@name{#2}
  \spath@array@init
  #1
  \expandafter\global\expandafter\let\csname spath@array@\spath@path@name @\the\value{spath@array}\endcsname=\spath@tmppath
  \expandafter\xdef\csname spath@array@\spath@path@name @length\endcsname{\the\value{spath@array}}
  \endgroup
}

\def\spath@prepare#1#2{
  \let\spath@tmppath=\pgfutil@empty
  \let\spath@tmppathrev=\pgfutil@empty
  \begingroup
  \setcounter{spath@array}{0}
  \setcounter{spath@length}{0}
  \def\spath@path@name{#2}
  \spath@prepare@init
  #1
   \expandafter\global\expandafter\let\csname spath@array@\spath@path@name @\the\value{spath@array}\endcsname=\spath@tmppath
   \g@addbefore@macro\spath@tmppathrev\pgfsyssoftpath@movetotoken
   \expandafter\global\expandafter\let\csname spath@array@\spath@path@name @\the\value{spath@array}@rev\endcsname=\spath@tmppathrev
   \expandafter\xdef\csname spath@array@\spath@path@name @\the\value{spath@array}@end\endcsname{\noexpand\pgfqpoint{\the\spath@ex}{\the\spath@ey}}
  \expandafter\xdef\csname spath@array@\spath@path@name @length\endcsname{\the\value{spath@array}}
  \expandafter\xdef\csname spath@array@\spath@path@name @\the\value{spath@array}@length\endcsname{\the\value{spath@length}}
   \expandafter\global\expandafter\let\csname spath@array@\spath@path@name @\the\value{spath@array}@first\endcsname=\spath@first
   \expandafter\global\expandafter\let\csname spath@array@\spath@path@name @\the\value{spath@array}@last\endcsname=\spath@last
  \endgroup
}

%    \end{macrocode}
% \iffalse
%</package>
% \fi
%
% \Finale

\endinput
