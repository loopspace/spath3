\RequirePackage{shellesc}
\immediate\write18{tex spath3_code.dtx}
\documentclass{ltxdoc}
\usepackage[T1]{fontenc}
\usepackage{trace}
\usepackage{lmodern}
\usepackage{morefloats}
\usepackage{tikz}
\usetikzlibrary{spath3,hobby,patterns,intersections}
\usepackage[numbered]{hypdoc}
\definecolor{lstbgcolor}{rgb}{0.9,0.9,0.9} 
 
\usepackage{listings}
\lstloadlanguages{[LaTeX]TeX}
\lstset{breakatwhitespace=true,breaklines=true,language=TeX}
 
\usepackage{fancyvrb}

\newenvironment{example}
  {\VerbatimEnvironment
   \begin{VerbatimOut}{example.out}}
  {\end{VerbatimOut}
   \begin{center}
   \setlength{\parindent}{0pt}
   \fbox{\begin{minipage}{.9\linewidth}
     \lstset{breakatwhitespace=true,breaklines=true,language=TeX,basicstyle=\small}
     \lstinputlisting[]{example.out}
   \end{minipage}}

   \fbox{\begin{minipage}{.9\linewidth}
     \centering
     \input{example.out}
   \end{minipage}}
\end{center}
}

\providecommand*{\url}{\texttt}
\GetFileInfo{spath3.sty}

\title{The \textsf{spath3} Package: Documentation}
\author{Andrew Stacey \\ \url{loopspace@mathforge.org}}
  \date{\fileversion~from \filedate}

  \begin{document}

  \maketitle

  \section{Introduction}

  The \texttt{spath3} package was originally designed as a low-level package for manipulating the \emph{soft paths} defined by PGF/TikZ.
  Soft paths form one stage of the stack of translations between what the author writes in the \texttt{tikzpicture} environments in their \LaTeX\ document and what is eventually written to the output file.
  Most of the complicated processing has been done by the time a soft path is constructed, but it is still very definitely a \TeX\ object and there has not, for example, been any consideration as to what the eventual output file format is (such as PDF, DVI, or SVG).
  So it is very amenable to being modified at this stage and this package provides a set of routines for doing so.

  The original purpose was to provide a common core on which other packages would be built.
  Indeed, the packages \texttt{calligraphy}, \texttt{knots}, and \texttt{penrose} all use this package.
  However, over time I've found myself wanting to use the routines of this package at a higher level and so have designed some user-level interfaces.
  This document documents those.

  \section{TikZ Keys}

\begin{verbatim}
\usetikzlibrary{spath3}
\end{verbatim}

The \texttt{spath3} TikZ library defines a set of keys that can be issued to muck about with soft paths.


\DescribeMacro{save spath}\texttt{save spath=<name>}

Saves the current path to an \texttt{spath} object called \texttt{<name>}.
This delays until the path is fully constructed so can be issued in the options to the main command.

Spaths constructed this way are global.

\begin{example}
\begin{tikzpicture}
\fill[green] (2,0) circle[radius=2pt] (2,2) circle[radius=2pt];
\draw[save spath=mypath] (2,0) to[out=0,in=180] (2,2);
\end{tikzpicture}
\end{example}

\DescribeMacro{restore spath}\texttt{restore spath=<name>}

Restores a previously saved \texttt{spath} to the current path.
This happens immediately so can be issued in the options to the main command and then the path can be extended with normal drawing commands.

One thing should be noted about transformations.
By the time a soft path is built, all available transformations have been applied.
This means that when re-inserting a soft path back into a high level command (such as \Verb+\draw+), the effect of existing transformations might produce some confusing effects.
When restoring a path then the library tries to set the last point correctly, but depending on how this is used it can result in transformations being applied twice.

\begin{example}
\begin{tikzpicture}
\fill[green] (2,0) circle[radius=2pt] (2,2) circle[radius=2pt];
\draw[restore spath=mypath];
\end{tikzpicture}

\begin{tikzpicture}
\fill[green] (2,0) circle[radius=2pt] (2,2) circle[radius=2pt] +(2,0) circle[radius=2pt];
\draw[restore spath=mypath] -- +(2,0);
\end{tikzpicture}
\end{example}


\DescribeMacro{insert spath}\texttt{insert spath=<name>}

This inserts an \texttt{spath} into the path at the current point, it is therefore more suited to being used part way through a path construction.
In a sense, it is a little like a \texttt{pic} in that it enables the user to construct a path segment early to be reused at various places.

The path is \emph{welded} on to the current path, meaning that it starts from the current point and there is no \texttt{move}.

\begin{example}
\begin{tikzpicture}
\fill[green] (0,0) circle[radius=2pt] (1,0) circle[radius=2pt] ++(0,2) circle[radius=2pt] ++(2,0) circle[radius=2pt];
\draw (0,0) -- (1,0) [insert spath=mypath] -- ++(2,0);
\end{tikzpicture}
\end{example}

\DescribeMacro{reverse spath}\texttt{reverse spath=<name>}

Reverses the path in the named spath object.
This doesn't do any actually drawing.
The effect is local.

\DescribeMacro{insert reverse spath}\texttt{insert reverse spath=<name>}

Like \texttt{insert spath} except that the inserted path is reversed first.

\DescribeMacro{append spath}\texttt{append spath=<name>}

Like \texttt{insert spath} except that is doesn't move the inserted path.

\DescribeMacro{shorten spath at end}
\DescribeMacro{shorten spath at start}
\DescribeMacro{shorten spath at both ends}
\texttt{shorten spath at end={<name>}{<dimen>}}

Shortens the spath by the dimension from one or both ends.
This shortens the path along its length so that it is guaranteed that the shorter path traces a subset of the original path.
For B\'ezier curves, however, the amount of shortening is approximate and is better for shorter distances.
If wanting to shorten by a large amount it is better to shorten by a small amount a number of times.

\DescribeMacro{globalise spath}\texttt{globalise spath=<name>}

Converts an spath object to a global one.

\DescribeMacro{translate spath}\texttt{translate spath={<name>}{<x-dimen>}{<y-dimen>}}

Translates the spath by the given dimensions.

\DescribeMacro{transform spath}\texttt{transform spath={<name>}{<transformations>}}

This applies the transformation to the spath.
The transformation is processed by TikZ so should consist of TikZ-level transformations such as \Verb+shift={(2,2)}+.

\DescribeMacro{split at self intersections}\texttt{split at self intersections={<prefix>}{<splits>}}

This splits the current path into segments at the points where it intersects itself.
The result is a set of spath objects with names \Verb+<prefix> <index>+, with \Verb+<index>+ running from \(1\) to \Verb+\spathselfintersectioncount+.
The \Verb+<splits>+ argument can be used to control which intersections are actually used.
It is a list of numbers from \(0\) to \(3\).
A \(0\) means that at an intersection then both parts of the path are split.
A \(1\) means to not split the first, a \(2\) to not split the second, a \(3\) means don't split either.
(Actually, both are split but then conditionally rejoined.)

\begin{example}
\begin{tikzpicture}[
  use Hobby shortcut,
]

\fill[red!50!white] (-3,3) rectangle (3,-3);
\fill[pattern=bricks, pattern color=white] (-3,3) rectangle (3,-3);

\path
[
  split at self intersections={trefoil}{121}
]
([closed]90:2) foreach \k in {1,...,3} { .. (-30+\k*240:.5) .. (90+\k*240:2) };

\foreach \k in {1,..., \spathselfintersectioncount} {
  \tikzset{shorten spath at both ends={trefoil \k}{5pt}}
  \draw[line width=\k pt, restore spath=trefoil \k];
}

\end{tikzpicture}
\end{example}

\DescribeMacro{split at intersections}\texttt{split at intersections={<first>}{<second>}{<first prefix>}{<second prefix>}{<splits>}}

This splits a pair of paths at their mutual intersections.

\begin{example}
\begin{tikzpicture}[
  use Hobby shortcut,
]

\fill[red!50!white] (-.5,-.5) rectangle (8.5,1.5);
\fill[pattern=bricks, pattern color=white] (-.5,-.5) rectangle (8.5,1.5);

\path[save spath=pathA] (0,0) to[out=0,in=180] ++(2,1) to[out=0,in=180] ++(2,-1)  to[out=0,in=180] ++(2,1)  to[out=0,in=180] ++(2,-1);
\path[save spath=pathB] (0,1) to[out=0,in=180] ++(2,-1) to[out=0,in=180] ++(2,1)  to[out=0,in=180] ++(2,-1)  to[out=0,in=180] ++(2,1);

\tikzset{split at intersections={pathA}{pathB}{pathA}{pathB}{1212}}

\foreach \k in {1,..., \spathfirstintersectioncount} {
  \tikzset{shorten spath at both ends={pathA \k}{5pt}}
  \draw[green!50!black, ultra thick, restore spath=pathA \k];
}

\foreach \k in {1,..., \spathsecondintersectioncount} {
  \tikzset{shorten spath at both ends={pathB \k}{5pt}}
  \draw[blue, ultra thick, restore spath=pathB \k];
}

\end{tikzpicture}
\end{example}

\section{Commands}

In all the following commands, using a star makes it global.

\DescribeMacro{\MakeSPath} \Verb+\MakeSPath*{<name>}{<path>}+

Creates an spath object from a path.

\DescribeMacro{\CloneSPath} \Verb+\CloneSPath*{<source>}{<target>}+

Clones the \Verb+source+ into the \Verb+target+.

\DescribeMacro{\SPathInfo}
\DescribeMacro{\SPathInfoInto}
\Verb+\SPathInfo{<name>}{<property>}+
\Verb+\SPathInfoInto{<name>}{<property>}{<macro>}+

Extract the \Verb+property+ from the named spath, the second version stores it in the given macro.

\DescribeMacro{\SPathPrepare}
\Verb+\SPathPrepare{<name>}+

Calculates all of the information associated to an spath.

\DescribeMacro{\SPathShow}
\Verb+\SPathShow{<name>}+

Dumps the information stored on the given spath to the log file.

\DescribeMacro{\SPathTranslate}
\DescribeMacro{\SPathTranslateInto}
\Verb+\SpathTranslate{<name>}{<x dimen>}{<y dimen>}+
\Verb+\SpathTranslateInto{<name>}{<name>}{<x dimen>}{<y dimen>}+

Translates the given spath by the given dimensions.
The first version modifies the given spath, the second clones it into the a second spath first.

\DescribeMacro{\SPathScale}
\DescribeMacro{\SPathScaleInto}
\Verb+\SpathScale{<name>}{<x dimen>}{<y dimen>}+
\Verb+\SpathScaleInto{<name>}{<name>}{<x dimen>}{<y dimen>}+

Scales the given spath by the given dimensions.
The first version modifies the given spath, the second clones it into the a second spath first.

\DescribeMacro{\SPathWeld}
\DescribeMacro{\SPathWeldInto}
\Verb+\SPathWeld*{<first>}{<second>}+
\Verb+\SPathWeldInto*{<first>}{<target>}{<second>}+

Welds the second spath onto the first.
This involves translating the second path so that it starts where the second ends, and removes the intervening move.

\DescribeMacro{\SPathReverse}
\Verb+\SPathReverse*{<name>}+

Reverses the spath.

\DescribeMacro{\SPathClose}
\Verb+\SPathClose*{<name>}+

Adds a closepath to the end of the spath.

\DescribeMacro{\SPathOpen}
\Verb+\SPathOpen*{<name>}+

Removes all closepaths from the spath, replacing them by lines if appropriate.

\DescribeMacro{\MakeSPathList}
\Verb+\MakeSPathList*{<path>}{<name>}+

Splits a path into a named list of segments.

\DescribeMacro{\SPathListPrepare}
\Verb+\SPathListPrepare*{<name>}+

Calculates all of the information for all of the spaths in the named list.

\DescribeMacro{\SPathListOpen}
\Verb+\SPathListOpen*{<name>}+

Opens all the spaths in the named list.

\end{document}
